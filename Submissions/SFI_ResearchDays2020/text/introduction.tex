
%%%%%%%%%%%%%%%%%%%%%%%%%%%%%%%%%%%%%%%%%%%%%%
%MOTIVATION

Lending relationships are key drivers for both the availability and pricing of credit \citep{Bharath2007, Ioannidou2010}. While lending relationships generally benefit borrowers, they also expose them to adverse shocks on the lender level, providing a transmission mechanism between the financial and real sector \citep{Ivashina2010, chodorow2014}. 

While it is well understood that lending relationships have a large impact on lending, how these relationships are formed, and how banks and borrowers actually match up is a much less studied topic. In this paper, we take a step towards answering this important question by studying the role of commercial bankers in matching banks and borrowers. 

%Summary
We find that personal relationships between bankers and firms are a key factor in matching lenders to borrowers. After a commercial banker switches from one bank to another, the likelihood of a relationship initiation by this new employing lender to the firmer borrower increases by a factor of 4 compared to the unconditional sample average. 

These results hold under tight controls including borrower-bank fixed effects, meaning that our results are driven by with-borrower-bank changes in having a personal relationship through a banker. Bank-year and borrower-year fixed effects control for lender and borrower time specific trends in initiation new relationships, ruling out a wide range of alternative explanations for our findings, such as a lender expanding and both hiring additional employees and initiating new lending relationships. 


Importantly, these initiations go beyond straight loan contracts. We find that after a banker with a personal tie to a borrower moves, the former borrower also issues new bonds and seasoned equity offerings with the new lender. 

While a large amount of business comes in the initial year of transition of the banker, borrowers continue issuing new bonds and loans in the following years.

Paper schwert bank borrower mtaching JF 2019:
\url{https://papers.ssrn.com/sol3/papers.cfm?abstract_id=2690490}

Gao et al punishment after client bankrupt (RFS conditional accept):
\url{https://papers.ssrn.com/sol3/papers.cfm?abstract_id=2865194}

Kleiner Paper on manager connections heling them find employment:
\url{https://papers.ssrn.com/sol3/papers.cfm?abstract_id=2954837}

wsj story, includes tangential reference of bank poaching a wells fargo banker and subsequently manage to move clients over. 
\url{https://www.wsj.com/articles/losing-450-000-in-three-days-hackers-trick-victims-into-big-wire-transfers-11582453800?mod=hp_lead_pos5}

%Theory

%Results

%Literature
