\documentclass[notes,11pt, aspectratio=169]{beamer}

\usepackage{pgfpages}
% These slides also contain speaker notes. You can print just the slides,
% just the notes, or both, depending on the setting below. Comment out the want
% you want.
\setbeameroption{hide notes} % Only slide
%\setbeameroption{show only notes} % Only notes
%\setbeameroption{show notes on second screen=right} % Both

\usepackage{helvet}
\usepackage[default]{lato}
\usepackage{array}


\usepackage{tikz}
\usepackage{verbatim}
\setbeamertemplate{note page}{\pagecolor{yellow!5}\insertnote}
\usetikzlibrary{positioning}
\usetikzlibrary{snakes}
\usetikzlibrary{calc}
\usetikzlibrary{arrows}
\usetikzlibrary{decorations.markings}
\usetikzlibrary{shapes.misc}
\usetikzlibrary{matrix,shapes,arrows,fit,tikzmark}
\usepackage{amsmath}
\usepackage{mathpazo}
\usepackage{hyperref}
\usepackage{lipsum}
\usepackage{multimedia}
\usepackage{graphicx}
\usepackage{multirow}
\usepackage{graphicx}
\usepackage{dcolumn}
\usepackage{booktabs,tabularx}
\usepackage{bbm}
\usepackage{hyperref}
\newcolumntype{d}[0]{D{.}{.}{5}}

\usepackage{changepage}
\usepackage{appendixnumberbeamer}
\newcommand{\beginbackup}{
   \newcounter{framenumbervorappendix}
   \setcounter{framenumbervorappendix}{\value{framenumber}}
   \setbeamertemplate{footline}
   {
     \leavevmode%
     \hline
     box{%
       \begin{beamercolorbox}[wd=\paperwidth,ht=2.25ex,dp=1ex,right]{footlinecolor}%
%         \insertframenumber  \hspace*{2ex} 
       \end{beamercolorbox}}%
     \vskip0pt%
   }
 }
\newcommand{\backupend}{
   \addtocounter{framenumbervorappendix}{-\value{framenumber}}
   \addtocounter{framenumber}{\value{framenumbervorappendix}} 
}


\usepackage{graphicx}
\usepackage[space]{grffile}
\usepackage{booktabs, fontawesome}

% These are my colors -- there are many like them, but these ones are mine.
\definecolor{blue}{RGB}{0,114,178}
\definecolor{red}{RGB}{213,94,0}
\definecolor{yellow}{RGB}{240,228,66}
\definecolor{green}{RGB}{0,158,115}

\hypersetup{
  colorlinks=false,
  linkbordercolor = {white},
  linkcolor = {blue}
}


%% I use a beige off white for my background
\definecolor{MyBackground}{RGB}{255,253,218}

%% Uncomment this if you want to change the background color to something else
%\setbeamercolor{background canvas}{bg=MyBackground}

%% Change the bg color to adjust your transition slide background color!
\newenvironment{transitionframe}{
  \setbeamercolor{background canvas}{bg=yellow}
  \begin{frame}}{
    \end{frame}
}

\setbeamercolor{frametitle}{fg=blue}
\setbeamercolor{title}{fg=black}
\setbeamertemplate{footline}[frame number]
\setbeamertemplate{navigation symbols}{} 
\setbeamertemplate{itemize items}{-}
\setbeamercolor{itemize item}{fg=blue}
\setbeamercolor{itemize subitem}{fg=blue}
\setbeamercolor{enumerate item}{fg=blue}
\setbeamercolor{enumerate subitem}{fg=blue}
\setbeamercolor{button}{bg=MyBackground,fg=blue,}

% If you like road maps, rather than having clutter at the top, have a roadmap show up at the end of each section 
% (and after your introduction)
% Uncomment this is if you want the roadmap!
% \AtBeginSection[]
% {
%    \begin{frame}
%        \frametitle{Roadmap of Talk}
%        \tableofcontents[currentsection]
%    \end{frame}
% }
\setbeamercolor{section in toc}{fg=blue}
\setbeamercolor{subsection in toc}{fg=red}
\setbeamersize{text margin left=1em,text margin right=1em} 

\newenvironment{wideitemize}{\itemize\addtolength{\itemsep}{10pt}  \setlength\itemsep{.5em}}{\enditemize}

\title[]{How do borrowers find their banks? \\ \vspace{.2cm} The value of individuals in bank relationship formation}
\author[PGP]{Marco Ceccarelli \inst{1}, Christoph Herpfer \inst{2}, Steven Ongena \inst{1}}
\institute[FRBNY]{\small{\begin{tabular}{c c c}
 \inst{1} Swiss Finance Institute and UZH & \hspace*{0.01cm}  & \inst{2} Emory University, Goizueta Business School  \\ \\
        \vspace*{0.5cm}\includegraphics[scale=0.1]{./figures/sfi_logo}   & \hspace*{0.1cm}  &  \includegraphics[scale=0.20]{./figures/GBS_hz_280} \vspace{.2cm} \\ 
\multicolumn{3}{c}{Brown Bag Lunch Seminar} 
\end{tabular}}}

\date{March 9, 2020}


\begin{document}

%%% TIKZ STUFF
\tikzset{   
        every picture/.style={remember picture,baseline},
        every node/.style={anchor=base,align=center,outer sep=1.5pt},
        every path/.style={thick},
        }
\newcommand\marktopleft[1]{%
    \tikz[overlay,remember picture] 
        \node (marker-#1-a) at (-.3em,.3em) {};%
}
\newcommand\markbottomright[2]{%
    \tikz[overlay,remember picture] 
        \node (marker-#1-b) at (0em,0em) {};%
}
\tikzstyle{every picture}+=[remember picture] 
\tikzstyle{mybox} =[draw=black, very thick, rectangle, inner sep=10pt, inner ysep=20pt]
\tikzstyle{fancytitle} =[draw=black,fill=red, text=white]
%%%% END TIKZ STUFF

% Title Slide
\begin{frame}
  \maketitle
\end{frame}
\note[itemize]{
\item Today we are looking at the question of how borrowers find their banks or more specifically, of what role individual bankers play in this relationship
\item Joint work with Lausanne graduate Christoph, now at Emory, and Steven
\item Super-preliminary stage so all feedback is welcome 
}

%%%%%%%%%%%%%%%%%%%%%%%%%%%%%%%%%%%%%%%
% BIG PICTURE QUESTION AND MOTIVATION
%%%%%%%%%%%%%%%%%%%%%%%%%%%%%%%%%%%%%%%
\section{Big picture}
\begin{frame}{Anecdotal evidence}
\begin{columns}[c] % align columns
  \begin{column}{.4\textwidth}
  \resizebox{\textwidth}{!}{
    \includegraphics[scale=0.2]{figures/Thiam_hires.jpg}
  }
  \end{column}% 
\hfill%
  \begin{column}{.5\textwidth}
    \faQuoteLeft~ Star bankers are routinely subjected to rough treatment when they jump ship to a rival. [...] Credit Suisse took the war-on-talent to a whole new level when it hired private investigators to tail Iqbal Khan, the bank’s former wealth management chief, \textcolor{red}{fearing he might be poaching talent and stealing business leads} in the days before starting a new job at UBS Group AG, in Zurich.~\faQuoteRight \\ \vspace{.2cm} \hfill  (Fortune, October 1, 2019)
    %The move backfired in spectacular and tragic fashion with the death of a private investigator, the ouster of a top-ranked Credit Suisse official—the CEO’s top lieutenant—and the bank’s reputation in tatters.
  \end{column}%
\end{columns}
\end{frame}
\note[itemize]{
\item You certainly have heard about the scandal involving Iqbal Khan, Credit Suisse's former head of Wealth Management who had moved to rival bank UBS. Tidjane Thiam (pictured) resigned amid a power struggle which followed a spying scandal at Credit Suisse. 
\item If human capital is so important that it warrants spying on employees that switch from employer, it warrants understanding 
\item a) the reasons that make bankers leave their bank and
\item b) what happens once they arrive at their new employer - i.e., are the concerns of the old employer warranted?
}

%%%%%%%%%%%%%%%%%%%%%%%%%%%%%%%%%%%%%%%
% PREVIEW
%%%%%%%%%%%%%%%%%%%%%%%%%%%%%%%%%%%%%%%
\begin{frame}{Preview of findings}
\textcolor{red}{What makes it more likely that a banker switches employer?}
\begin{wideitemize}
    \onslide<2->{ 
      \item Having a larger portfolio of clients, especially when 
        \\ (a) the relationship with them is strong and 
        \\ (b) they are more likely to come over
      \item Also, banker's tenure \& how bank's client portfolio is divided amongst bankers matters
     }
\end{wideitemize} \vspace{.3cm}

\onslide<3->{ \textcolor{red}{Does the new bank profit from the banker switching?} }
\begin{wideitemize}
    \onslide<4->{ 
      \item Yes! The bank increases its borrower base by winning over clients known to the banker
      \item The new business that is brought over extends to syndicated lending and bond underwriting
     }
\end{wideitemize}  \vspace{.3cm}

\onslide<5->{\begin{center}  \textcolor{red}{Bankers play an important role in relationship lending. \\ \vspace{.1cm} Banks that acquire human capital profit by extending their borrower base.} 
\end{center} } 
\end{frame}
\note[itemize]{
\item Let me defer for one second the exposition of what data I use and how I identify the bankers that switch employers and let me give you a short preview of the findings that we have so far
\item obviously, wages also important but no data.
}

%%%%%%%%%%%%%%%%%%%%%%%%%%%%%%%%%%%%%%%
% LITERATURE OVERVIEW
%%%%%%%%%%%%%%%%%%%%%%%%%%%%%%%%%%%%%%%
\begin{frame}{Literature review I - Relationship lending}
\textcolor{red}{Relationship lending plays a key role for both banks and firms}
\begin{wideitemize}
    \onslide<2->{ 
     \item The ability of banks to create information about their borrowers is at the core of banking (e.g., Berger and Udell, JofB 1995; Diamond, REStud 1984; Petersen and Rajan, JF 1994) 
     \item The soft information of these relationships is concentrated in individual bankers (Liberti and Petersen, RCFS 2019; Karolyi, JF 2018)
     \item Banking relationships play a key role also for the firms, influencing loan conditions (Ioannidou and Ongena, JF 2010) and availability (Ongena and Smith, JFE 2001)
     }
\end{wideitemize} 

\onslide<3->{ \vspace{.2cm}
      \textcolor{red}{What drives the formation of relationships between firms and banks?} \\ 
      \begin{wideitemize} \item Schwert, JF 2018; Petersen and Rahan, QJE 1995; JF 2002;... look at the importance of bank characteristics for the formation of borrower relationships \end{wideitemize} }

\end{frame}
\note[itemize]{
  \item Relationship lending generates soft information that can be used instead of collateral or more generally speaking to monitor borrowers
  \item Berger and Udell, JofB 1995 find that borrowers with relationships pay lower interests and are less likely to pledge collateral 
  \item Petersen and Rajan, JF 1994;  empirically examines how ties between a firm and its creditors affect the availability and cost of funds to the firm ``Relationships are valuable and appear to operate more through quantities rather than prices.''
  \item Karoly JF 2018 - Use executive deaths to show that connections between banks and management make lending more likely
}

\begin{frame}{Literature review II - Role of bankers}
  \textcolor{red}{Bankers play an important role in the lending process}
  \begin{wideitemize}
    \onslide<2->{ 
    \item Bushman, Gao, Martin, Pacelli, 2019 find that bankers are important in determining loan characteristics, especially covenant design  
    \item Frattaroli, Herpfer, 2020 find that bankers help firms identify partners for strategic alliances 
    \item Gao, Kleiner, Pacelli, RFS 2020 find that bankers that structure poorly performing loans face disciplining consequences
    \item Herpfer, 2018 finds that strong relationships between a borrower and a banker significantly reduce interest rates }
  \end{wideitemize}

\onslide<3->{  \vspace{.2cm} \textcolor{red}{Our contribution:} Look at the \textcolor{red}{role of bankers} in \textcolor{red}{relationship lending}}
\end{frame}
\note[itemize]{
  \item Nascent literature that examines the role of bankers in the lending process
  \item Our contribution is to bring together the relationship lending literature and the more recent studies on the role of individual bankers and ask what the impact of individuals is on the formation of relationships between banks and their clients
}

%%%%%%%%%%%%%%%%%%%%%%%%%%%%%%%%%%%%%%%
% DATA AND METHODOLOGY
%%%%%%%%%%%%%%%%%%%%%%%%%%%%%%%%%%%%%%%
\section{Data \& methodology}
\begin{frame}[label=data_bankers]{Data I - Individual bankers}
  \begin{columns}[c] % align columns
  \begin{column}{.4\textwidth}
  \resizebox{\textwidth}{!}{
    \includegraphics[scale=0.45]{../../Writeup/figures/signature_well_formated.png}
  }
  \end{column}% 
\hfill%
  \begin{column}{.5\textwidth}
    \begin{wideitemize}
      \item Loans considered ``material events''\\ \faArrowRight~ firms must file loan contracts to SEC
      \onslide<2->{
      \item Scrape all 8-K, 10-K, and 10-Q filings and obtain loan information:
      \begin{wideitemize}
          \item \textcolor{red}{Bank Name}
          \item Bank Role
          \item \textcolor{red}{Person Name}
          \item Person Title
        \end{wideitemize} }

      \onslide<3->{\item Obtain \textcolor{red}{personal relationships} between banker and clients \textcolor{red}{and} identify \textcolor{red}{bankers that switch} their employer.
      \item \hyperlink{appendix_bankers}{\beamergotobutton{Quality-check}} }
    \end{wideitemize}
   
  \end{column}%
\end{columns}
\end{frame}
\note[itemize]{
\item An 8K can be any sort of announcement of significant corporate information. It's like a press release by the company. A 10K is a formal annual filing that contains the annual financial statements and lots of other information.
}

%%%%%%%%%%%%%%%%%%%%%%%%%%%%%%%%%%%%%%%%%%%%%%%%%%%%%%%%%%%%%%%%%%%%%%
%%%%%%%%%%%%%%%%%%%%%%%%%%%%%%%%%%%%%%%%%%%%%%%%%%%%%%%%%%%%%%%%%%%%%%
\begin{transitionframe}
  \begin{center}
    \Huge Part I \\ ~ \\ What makes a banker more likely \\ to switch?
  \end{center}
\end{transitionframe}
%%%%%%%%%%%%%%%%%%%%%%%%%%%%%%%%%%%%%%%%%%%%%%%%%%%%%%%%%%%%%%%%%%%%%%
%%%%%%%%%%%%%%%%%%%%%%%%%%%%%%%%%%%%%%%%%%%%%%%%%%%%%%%%%%%%%%%%%%%%%%

% EXAMPLE OF PERS-REL-ACQ AND INITIATION 
  % >> Show only pre-switch & come back at rel_acq later when you're talking about it!
\begin{frame}{Identifying when a banker switches: Banker Joe}
\resizebox{\textwidth}{!}{ \centering 
\begin{tabular}{ccccccc} %m{.1\linewidth}m{.1\linewidth}m{.2\linewidth}m{.3\linewidth}
  Yr & Bank & Deal & Old-Client-Portfolio & Pre-Switch & Total \#Deals & \#Clients - Mod \\ \midrule
  \marktopleft{a0}2000 & BofA & GE & - & 1 & 1 & 0\markbottomright{a0}{red}  \\
  2001 & BofA & Siemens & - & 1 & 2 & 0 \\
  \marktopleft{a1}\marktopleft{a2}2002 & BofA & Siemens & - & 1 & 3 & 1\markbottomright{a1}{red} \\ 
  \multicolumn{7}{c}{...} \\
  2005 & JPM & VW & GE, Siemens\markbottomright{a2}{red} & 0 & 1 & 0 \\
  2006 & JPM & - & GE, Siemens & 0 & 1 & 0 \\ 
  2007 & JPM & GE & GE, Siemens & 0 & 2 & 0 \\ 
  2008 & JPM & VW & GE, Siemens & 0 & 3 & 1 \\ 
  \multicolumn{7}{c}{...}  \\ \bottomrule \vspace{.2cm}
  
\end{tabular} } 
  \only<1>{~ \\ ~ \\ ~ \vfill}

  \only<2>{\centering 2000: Banker Joe closes hist 1st deal with GE \\ \faArrowRight~ the total \#deals is set to 1 \vfill \tikz[overlay,remember picture,inner sep=1pt]
\node[draw=red,rounded corners,fit=(marker-a0-a.north west) (marker-a0-b.south east)] {};}
  
  \only<3>{\centering 2002: Banker Joe closes the 2nd deal with Siemens \\ \faArrowRight~ has 1 client with a \textcolor{red}{moderately strong relationship} \vfill \tikz[overlay,remember picture,inner sep=1pt]
\node[draw=red,rounded corners,fit=(marker-a1-a.north west) (marker-a1-b.south east)] {};}
  \only<4>{\centering 2005: Banker Joe \textcolor{red}{switches from BofA to JPM} \\ \faArrowRight~ JPM acquires the personal relationships of banker Joe from his time at BofA \vfill \tikz[overlay,remember picture,inner sep=1pt]
\node[draw=red,rounded corners,fit=(marker-a2-a.north west) (marker-a2-b.south east)] {};}
\end{frame}

%%%%%%%%%%%%%%%%%%%%%%%%%%%%%%%%%%%
%% DATA - SUMMARY SATS
%%%%%%%%%%%%%%%%%%%%%%%%%%%%%%%%%%%
\begin{frame}{Data I - Deal volume and bank information}
We complement the SEC information with:
\begin{wideitemize}
  \item Syndicated loans from LPC Dealscan \\
  (loan characteristics and deals for which the algorithm gets no information)
  \item Bond underwriting and SEOs from CapitalIQ 
  \item Balance sheet information from Compustat 
\end{wideitemize} \vspace{.2cm}
\faArrowRight~ 20,000+ bankers that sign a total of 16,700 deals from 1996 to 2013 \\ \vspace{.2cm}
\faArrowRight~ Collapse data at the banker $\times$ bank $\times$ year level
 
\end{frame}
\note[itemize]{
  \item Some 2,000 different institutions in total
  \item These include also banks that are not matched to Compustat and probably not listed/quite small
  \item Timeframe not updated since the SEC changed the way you can retrieve data from its web-servers and now it has become very tricky to scrape data 
  \item Pre-Switch and Post-Switch are always 0 for the bankers that never switch
  \item Data includes 16,700 deals with on average 5.3 signatures \faArrowRight~ ca. 90,000 observations
  \item Why collapse? Because you're interested in finding out which characteristics of the bank-banker relationship make a banker more likely to switch
}

%%%%%%%%%%%%%%%%%%%%%%%%%%
\begin{frame}{Data I - Summary statistics}
\makebox[\linewidth][c]{ \centering
  \begin{tabular}{lcccccc}
                      &           N&         p25&        mean&         p50&         p75&          sd\\
\midrule
 \marktopleft{a1}Ever-Switch (\%)    &      49,998&        0.00&       24.70&        0.00&        0.00&       43.13\\
Pre-Switch (\%)     &      49,998&        0.00&        7.86&        0.00&        0.00&       26.92\markbottomright{a1}{red}\\
Total \#Deals       &      49,998&        2.00&        7.90&        5.00&        9.00&       11.02\\
\marktopleft{a3}\#Clients - Weak Rel&      49,998&        0.00&        1.00&        1.00&        1.00&        1.14\\
\#Clients - Moderate Rel&      49,998&        0.00&        0.68&        0.00&        1.00&        1.13\\
\#Clients - Strong Rel&      49,998&        0.00&        0.06&        0.00&        0.00&        0.38\markbottomright{a3}{red}\\ 
\marktopleft{a4}\#Clients - Single Contact&      49,998&        0.00&        0.90&        1.00&        1.00&        1.20\\
\#Clients - Multiple Contact&      49,998&        0.00&        0.83&        1.00&        1.00&        1.25\markbottomright{a4}{red}\\ \marktopleft{a2}Tenure Current      &      49,998&        1.00&        3.21&        2.00&        4.00&        3.05\\
Tenure Max          &      49,998&        2.00&        5.45&        4.00&        8.00&        4.17\markbottomright{a2}{red}\\
\bottomrule \vspace{.2cm}
  \end{tabular} }

  \only<1>{~\hfill \\ ~\hfill \\ ~\hfill }
  \only<2>{\centering Indicators for bankers that switch from one bank to the other \\ ~\hfill \\ ~\hfill 
    \tikz[overlay,remember picture,inner sep=1pt] \node[draw=red,rounded corners,fit=(marker-a1-a.north west) (marker-a1-b.south east)] {};}
   \only<3>{\centering Weak relationship - only 1 deal with client \\ Moderate relationship - 2-5 deals \\ Strong relationship - 6 or more deals \tikz[overlay,remember picture,inner sep=1pt]
\node[draw=red,rounded corners,fit=(marker-a3-a.north west) (marker-a3-b.south east)] {};}
  \only<4>{\centering Banker is single contact for clients within bank  \\ Clients have contact with multiple bankers within same bank \\ ~\hfill \tikz[overlay,remember picture,inner sep=1pt]
\node[draw=red,rounded corners,fit=(marker-a4-a.north west) (marker-a4-b.south east)] {};}
\only<5>{\centering Running and maximum no. of years that a banker spends at a bank \\ ~\hfill \\ ~\hfill  \tikz[overlay,remember picture,inner sep=1pt] \node[draw=red,rounded corners,fit=(marker-a2-a.north west) (marker-a2-b.south east)] {};}
\end{frame}
\note[itemize]{
  \item Summary stats at the banker - bank - year level
  \item Number of clients with whom the banker closed only one deal / 2-5 deals / 6+ deals over a given year (!) at a given bank
}
%%%%%%%%%%%%%%%%%%%%%%%%%%%%%%%%%%%%%%%
% MAIN FINDINGS 1 - CLIENT PORTFOLIO
%%%%%%%%%%%%%%%%%%%%%%%%%%%%%%%%%%%%%%%
\section{Main Findings}
\begin{frame}{Finding I - What makes a banker more likely to switch?}
 \begin{columns}[c] % align columns
  \begin{column}{.6\textwidth}
        \resizebox{\textwidth}{!}{ \centering 
      \begin{tabular*}{\hsize}{@{\hskip\tabcolsep\extracolsep\fill}l*{2}{c}} \toprule 
        Dep. variable:  &\multicolumn{2}{c}{Pre-Switch Indicator (\%)}  \\ \midrule
      \marktopleft{a1}Total \#Deals   &     0.22***&     0.21*** \\
                      &   (4.73)   &    (4.71)\markbottomright{a1}{red}\\
      \#Clients - Weak Rel&     0.28   &     0.38*\\
                      &   (1.39)   &   (1.86)   \\
      \marktopleft{a2}\#Clients - Moderate Rel &     0.90***&     0.94***\\
                      &   (4.56)   &   (4.68)\markbottomright{a2}{red}\\
      \#Clients - Strong Rel&     0.30   &     0.44 \\
                      &         (0.42)   &   (0.60) \\
      \midrule Observations    &   43,233   &   43,233 \\
      R-squared       &     0.24   &     0.31 \\
      \midrule Year and Bank FE &        Yes   &       No   \\
      Bank-Year FE    &          No   &      Yes  \\
      \bottomrule
      \end{tabular*} 
  } \end{column}% 
\hfill%
  \begin{column}{.3\textwidth}
     \only<2>{\centering  Bankers that \\ \textcolor{red}{close more deals} are \\ more likely to switch \\ \vspace{.2cm} Closing an extra deal increases the probability of switching by 22bps \\ (3\% of uncond. mean)}
     \only<3>{\centering  This is especially true for bankers that have many \textcolor{red}{moderately strong relationships}, i.e., that close 2-5 deals with a given client} 
    \only<4>{\centering {\Large Not only quantity of client portfolio, but also the \\ \vspace{.1cm} \textcolor{red}{quality matters}} 
  }
  \end{column}%
\end{columns}
\only<2>{\tikz[overlay,remember picture,inner sep=1pt] \node[draw=red,rounded corners,fit=(marker-a1-a.north west) (marker-a1-b.south east)] {};}
\only<3>{\tikz[overlay,remember picture,inner sep=1pt] \node[draw=red,rounded corners,fit=(marker-a2-a.north west) (marker-a2-b.south east)] {};}
\end{frame}
\note[itemize]{
  \item To give a sense of the economic significance of the result, an additional moderate client increases the proba of ever switching by almost 1\% (19\% of the sample ever switches) 
  \item Standard errors clustered at the bank and banker level
  }

\begin{frame}[label=finding1]{Finding I - What makes a banker more likely to switch?}
 \begin{columns}[c] % align columns
  \begin{column}{.6\textwidth}
        \resizebox{\textwidth}{!}{ \centering 
      \begin{tabular*}{\hsize}{@{\hskip\tabcolsep\extracolsep\fill}l*{2}{c}} \toprule 
        Dep. variable:  &\multicolumn{2}{c}{Pre-Switch Indicator (\%)}  \\ \midrule
Total \#Deals   &    0.22***&     0.22***\\
                &   (5.00)   &   (4.95)   \\ 
\marktopleft{a1}\#Clients - Single Contact &  0.93***&     1.00***\\
               &   (4.08)   &   (4.12)\markbottomright{a1}{red}\\
\#Clients - Multiple Contact&   0.25   &     0.34*  \\
             &   (1.37)   &   (1.88)   \\
\midrule 
Observations    &  43,233   &   43,233   \\
R-squared       & 0.24   &     0.31   \\
\midrule Year and Bank FE &   Yes   &       No   \\
Bank-Year FE    &    No   &      Yes   \\
\bottomrule
      \end{tabular*} 
  } \end{column}% 
\hfill%
  \begin{column}{.3\textwidth}
     \only<2>{\centering Bankers that have more clients for which they are the \textcolor{red}{single contact} at the bank are more likely to switch. \\ ~ \\}
     \only<3>{\centering {\Large It is important to have clients \\ that you are \\ \vspace{.1cm} \textcolor{red}{likely to bring over} }
       \\  \hfill  \hyperlink{appendix_key}{\beamergotobutton{Key}} 
       \\  \hfill  \hyperlink{appendix_tenure}{\beamergotobutton{Tenure}}}
 \end{column}%
\end{columns}
\only<2>{\tikz[overlay,remember picture,inner sep=1pt] \node[draw=red,rounded corners,fit=(marker-a1-a.north west) (marker-a1-b.south east)] {};}
\end{frame}

%%%%%%%%%%%%%%%%%%%%%%%%%%%%%%%%%%%%%%%%%%%%%%%%%%%%%%%%%%%%%%%%%%%%%%
%%%%%%%%%%%%%%%%%%%%%%%%%%%%%%%%%%%%%%%%%%%%%%%%%%%%%%%%%%%%%%%%%%%%%%
\section{Main finding 2 - Initiation and deal volume}
\begin{transitionframe}
  \begin{center}
    \Huge Part II \\ ~ \\ Does the new bank profit \\ from the switch?
  \end{center}
\end{transitionframe}
%%%%%%%%%%%%%%%%%%%%%%%%%%%%%%%%%%%%%%%%%%%%%%%%%%%%%%%%%%%%%%%%%%%%%%
%%%%%%%%%%%%%%%%%%%%%%%%%%%%%%%%%%%%%%%%%%%%%%%%%%%%%%%%%%%%%%%%%%%%%%

%%%%%%%%%%%%%%%%%%%%%%%%%%%%%%%%%%%%%%%
% EXAMPLE 2 - INITIATION
%%%%%%%%%%%%%%%%%%%%%%%%%%%%%%%%%%%%%%%
\begin{frame}{Identifying relationship initiations: Banker Joe - Take 2}
\resizebox{\textwidth}{!}{%
 \centering 
\begin{tabular*}{\hsize}{@{\hskip\tabcolsep\extracolsep\fill}c*{5}{c}} 
  Yr & Bank & Deal & Old-Client-Portfolio & Initiation & Rel\_acquired \\ \midrule
  \marktopleft{a1}2000 & BofA & GE & - & 1 & 0 \\
  2001 & BofA & Siemens & - & 1 &  0\markbottomright{a1}{red}\\
  2002 & BofA & Siemens & - & 0 & 0 \\ 
  \multicolumn{6}{c}{...} \\
  \marktopleft{a2}2005 & JPM & VW & GE, Siemens & 1 & 0\markbottomright{a2}{red} \\
  2006 & JPM & - & GE, Siemens & 0 & 0 \\ 
  \marktopleft{a3}2007 & JPM & GE & GE, Siemens &  1 & 1\markbottomright{a3}{red} \\ 
  2008 & JPM & VW & GE, Siemens & 0 & 0 \\ 
  \multicolumn{6}{c}{...}  \\ \bottomrule \vspace{.2cm}
\end{tabular*} %
}  \hfill 
  \only<1>{~ \\ ~ \\ ~ \vfill}
  \only<2>{\centering \hfill \bigskip 2000 and 2001: BofA \textcolor{red}{initiates a new relationship} with GE and Siemens \vfill \tikz[overlay,remember picture,inner sep=1pt]
\node[draw=red,rounded corners,fit=(marker-a1-a.north west) (marker-a1-b.south east)] {};} 
  \only<3>{\centering 2005: Banker Joe \textcolor{red}{switches from BofA to JPM} \\ \faArrowRight~ JPM acquires the personal relationships of banker Joe from his time at BofA \vfill \tikz[overlay,remember picture,inner sep=1pt]
\node[draw=red,rounded corners,fit=(marker-a2-a.north west) (marker-a2-b.south east)] {};}
  \only<4>{\centering 2007: JPM initiates a new relationship with GE, \\ one of \textcolor{red}{Banker Joe's old clients} \vfill \tikz[overlay,remember picture,inner sep=1pt]
\node[draw=red,rounded corners,fit=(marker-a3-a.north west) (marker-a3-b.south east)] {};}
\end{frame}

\note[itemize]{
  \item Introduce 2 new variables, the first initiation takes the value of one the first time that a bank initiates contact with a new firm
  \item Initiation covers both ``organic'' growth and the clients that are brought over by bankers
  \item Relationship acquired, takes the value of 1 for all deals that the new bank makes with the old clients of banker Joe
}

%%%%%%%%%%%%%%%%%%%%%%%%%%%%%%%%%%%%%%%
% DATA 2 - INITIATION SUMMARY STATS
%%%%%%%%%%%%%%%%%%%%%%%%%%%%%%%%%%%%%%%
\begin{frame}{Data II - Summary statistics}
\makebox[\linewidth][c] {\centering
  \begin{tabular}{lcccccc}
                        &           N&         p25&        mean&         p50&         p75&          sd\\
\midrule
\marktopleft{a1}Initiation\_strict (\%)&     972,090&        0.00&        4.74&        0.00&        0.00&       21.25\markbottomright{a1}{red}\\
\marktopleft{a2}Initiation (\%)     &     972,090&        0.00&        5.19&        0.00&        0.00&       22.19\markbottomright{a2}{red}\\
\marktopleft{a3}Rel\_acq (\%)       &     972,090&        0.00&        2.93&        0.00&        0.00&       16.86\markbottomright{a3}{red}\\
\marktopleft{a4}Rel\_acq\(^{5yr}\) (\%)&     958,303&        0.00&        1.53&        0.00&        0.00&       12.28\\
Rel\_acq\(^{abs}\) (\%)&     946,223&        0.00&        0.27&        0.00&        0.00&        5.23\markbottomright{a4}{red}\\
\marktopleft{a5}Volume - All deals  &     972,090&        0.00&       75.86&        0.00&        0.00&      806.00\markbottomright{a5}{red}\\
\marktopleft{a6}Volume - Bonds      &     972,090&        0.00&       25.61&        0.00&        0.00&      376.80\\
Volume - SEOs       &     972,090&        0.00&        5.15&        0.00&        0.00&      139.88\\
Volume - Synd. Loans&     972,090&        0.00&       38.25&        0.00&        0.00&      490.65\markbottomright{a6}{red}\\
\bottomrule \vspace{.2cm}
  \end{tabular}
}
\centering 
\only<1>{Collapse dataset at the \textcolor{red}{bank $\times$ firm $\times$ year} level \& add bank-firm deals \\ (loans, bonds, and SEOs) w/o banker information \\ \vspace{.1cm}~ \faArrowRight~ Total of 50k loans, 25k bonds, and 13k SEOs} %(some deals counted multiple times if they are closed jointly by more than one bank)
\only<2>{Initiation\_strict identifies \textcolor{red}{1st time interaction} between a bank and a firm \\~  \\ \vspace{.1cm}~ \tikz[overlay,remember picture,inner sep=1pt] \node[draw=red,rounded corners,fit=(marker-a1-a.north west) (marker-a1-b.south east)] {};}
\only<3>{Initiation also includes deals with \textcolor{red}{stale clients} (no deal in more than 5yrs) \\~  \\ \vspace{.1cm}~ \tikz[overlay,remember picture,inner sep=1pt] \node[draw=red,rounded corners,fit=(marker-a2-a.north west) (marker-a2-b.south east)] {};}
\only<4>{Rel\_acq takes the value of 1 for \textcolor{red}{all pairs of new\_bank$\times$old\_client$\times$yr}, \\ for all years after the switch  \\ \vspace{.1cm}~ \tikz[overlay,remember picture,inner sep=1pt] \node[draw=red,rounded corners,fit=(marker-a3-a.north west) (marker-a3-b.south east)] {};}
\only<5>{Rel\_acq${^{5yr}}$ and Rel\_acq$^{abs}$ take the value of 1, \\ for \textcolor{red}{5-yrs and 1-yr after the switch} and set to missing afterwards  \\ \vspace{.1cm}~ \tikz[overlay,remember picture,inner sep=1pt] \node[draw=red,rounded corners,fit=(marker-a4-a.north west) (marker-a4-b.south east)] {};}
 \only<6>{Volume - All deals is \textcolor{red}{the sum of all deals} (loans, bonds, SEOs) that \\ a bank closes with a borrower within a year (in USDmm)  \\ \vspace{.1cm}~ \tikz[overlay,remember picture,inner sep=1pt] \node[draw=red,rounded corners,fit=(marker-a5-a.north west) (marker-a5-b.south east)] {};}
  \only<7>{Volume of deals that a bank closes during a year \textcolor{red}{by deal type} (UDSmm) \\~  \\ \vspace{.1cm}~  \tikz[overlay,remember picture,inner sep=1pt] \node[draw=red,rounded corners,fit=(marker-a6-a.north west) (marker-a6-b.south east)] {};}
\end{frame}

\note[itemize]{
\item The deals don't sum up because the data is at the banker-bank level, i.e., a deal can be counted multiple times at different banks 
\item The median is 0 all the time because you fill in the gaps and treat years with no deals as explicit zero deals
  \item Conditional on having a deal, the size of bonds (26k) is 1.5x larger than that of synd loans (50k)
  \item SEOs are the smallest of the lot (12.8k)
}

%%%%%%%%%%%%%%%%%%%%%%%%%%%%%%%%%%%%%%%
% MAIN FINDINGS 2.1 - INITIATION
%%%%%%%%%%%%%%%%%%%%%%%%%%%%%%%%%%%%%%%
\begin{frame}[label=finding2_init]{Finding IIa - Does the new bank profit from the switch?}
 \begin{columns}[c] % align columns
  \begin{column}{.65\textwidth}
    \resizebox{\textwidth}{!}{ \centering 
      \begin{tabular*}{\hsize}{@{\hskip\tabcolsep\extracolsep\fill}l*{4}{c}}
 \toprule Dep. variable: &\multicolumn{4}{c}{Initiation}                                  \\\cmidrule(lr){2-5}
                &\multicolumn{1}{c}{(1)}   &\multicolumn{1}{c}{(2)}   &\multicolumn{1}{c}{(3)}  &\multicolumn{1}{c}{(4)}  \\
\midrule
\marktopleft{a1}Rel\_acq        &     0.07** &     0.09** & 0.13*** &    0.14***   \\
                &   (2.37)   &   (2.38)  &      (3.58)  &   (3.80)\markbottomright{a1}{red}   \\
\midrule
Observations    &  861,444   &  861,444   &  861,444   &  861,444   \\
R-squared       &     0.03   &     0.08 &     0.10    &     0.42     \\
\midrule 
Year FE &      Yes   &      Yes   & Yes   &        No    \\
Firm FE         &      Yes   &       No   &    No   &      No     \\
\marktopleft{a2}Firm-Bank FE    &       No   &      Yes   &      Yes   &      Yes \\
Bank-Year FE    &       No   &       No   &      Yes  &    Yes \\ 
Firm-Bank FE    &       No   &      No   &      No  &      Yes\markbottomright{a2}{red}  \\ \bottomrule
\end{tabular*}%
  } \end{column}% 
\hfill%
  \begin{column}{.3\textwidth} 
  \only<2>{\centering  Probability of initiating contact with new firm \\ \textcolor{red}{increases after the \\ bank acquires a personal relationship} \\ \vspace{.2cm} This corresponds to \\ 1.5x - 3x the average unconditional probability of initiation \tikz[overlay,remember picture,inner sep=1pt] \node[draw=red,rounded corners,fit=(marker-a1-a.north west) (marker-a1-b.south east)] {}; }
  \only<3>{\centering Finding holds under \\ \textcolor{red}{tight FE structure}: \\ Bank $\times$ Year (supply), \\ Firm $\times$ Year (demand), and \\ Firm $\times$ Bank FEs (geographic proximity, compatible strategy etc.) \tikz[overlay,remember picture,inner sep=1pt] \node[draw=red,rounded corners,fit=(marker-a2-a.north west) (marker-a2-b.south east)] {};} 
  \only<4>{\centering Findings remain virtually unchanged when we: \\ \vspace{.1cm} 
  \begin{wideitemize} 
    \item use stricter definition of initiation \\ \hyperlink{appendix_initiation}{\beamergotobutton{Table}}  
    \item use different treatments (5-yrs \& absorptive) \\
    \hyperlink{appendix_initiation_treat}{\beamergotobutton{Table}}  
    \item drop first deal that banker signs at new bank \\
    \hyperlink{appendix_initiation_nofirst}{\beamergotobutton{Table}}  
    \end{wideitemize} }
  \only<5>{\Large \centering The bank increases its borrower base by \textcolor{red}{winning over clients known to the banker}}
      \end{column}%
\end{columns}
 \end{frame}

\note[itemize]{
 \item hammer down how strict this specification is: fill out sample, treat all relationships acquired as one from the moment the banker switches (explain why this is so conservative), include all relationships that are never brought over. Also, all other new clients that the bank acquires that are not in the bankers' portfolio count against us 
  \item Result holds when using the various types of relationship acquired 
  \item Result counts all years starting from the first time the banker appears at the new bank as rel\_acq=1
  \item Firm-bank-FE accounts for e.g. time-invariant bank-firm-pair characteristics such as geographic proximity, compatible corporate culture and strategy 
  \item Bank-year-FE account for changes in supply of lending at the bank level
  \item Firm-year-FE account for changes in demand at the firm 
  \item Standard errors are 2-way clustered around banks and bankers (adding time as the third dimension also doesn't change anything)
}

%%%%%%%%%%%%%%%%%%%%%%%%%%%%%%%%%%%%%%%
% MAIN FINDINGS 2.2 - DEAL VOLUME
%%%%%%%%%%%%%%%%%%%%%%%%%%%%%%%%%%%%%%%
\begin{frame}[label=finding2_vol]{Finding IIb - Does the new bank profit from the switch?}
 \begin{columns}[c] % align columns
  \begin{column}{.65\textwidth}
    \resizebox{\textwidth}{!}{ \centering 
\begin{tabular*}{\hsize}{@{\hskip\tabcolsep\extracolsep\fill}l*{4}{c}}
% \def\sym#1{\ifmmode^{#1}\else\(^{#1}\)\fi}
\toprule
Dep. variable: &\multicolumn{4}{c}{Log Deal Volume}                             \\\cmidrule(lr){2-5}
&\multicolumn{1}{c}{(1)}   &\multicolumn{1}{c}{(2)}   &\multicolumn{1}{c}{(3)}   &\multicolumn{1}{c}{(4)} \\
\midrule
\marktopleft{a1}Rel\_acq        &     0.65***&     0.72***&     0.61***&     0.30***\\
               &   (8.62)   &   (4.09)   &   (5.18)  &   (3.80)\markbottomright{a1}{red}  \\
\midrule
Observations    &  809,108   &  809,108   &  809,108   &  809,108   \\
R-squared       &     0.07   &     0.14   &     0.16   &     0.51   \\
\midrule
Year FE &      Yes   &      Yes   &      Yes  &      No  \\
Firm FE         &      Yes   &       No   &       No   &      No \\
\marktopleft{a2}Firm-Bank FE    &       No   &      Yes   &      Yes   &      Yes \\
Bank-Year FE    &       No   &       No   &      Yes  &    Yes \\ 
Firm-Bank FE    &       No   &      No   &      No  &      Yes\markbottomright{a2}{red}\\
\bottomrule
\end{tabular*}%
  } \end{column}% 
\hfill%
  \begin{column}{.3\textwidth} 
  \only<2>{\centering The banks that acquire relationships when bankers switch \textcolor{red}{close more deals} with the new clients \\ \vspace{.2cm} For the median deal \\ (USD 300mm, conditional on closing), this corresponds to an increase of USD 2.1mm \tikz[overlay,remember picture,inner sep=1pt] \node[draw=red,rounded corners,fit=(marker-a1-a.north west) (marker-a1-b.south east)] {};}
  \only<3>{\centering This holds under a \textcolor{red}{tight FE structure} \tikz[overlay,remember picture,inner sep=1pt] \node[draw=red,rounded corners,fit=(marker-a2-a.north west) (marker-a2-b.south east)] {};}
   \only<4>{\centering Findings remain virtually unchanged when: \\ \vspace{.1cm} 
      \begin{wideitemize} 
    \item using different treatments (5-yrs \& absorptive)  
     \\ \hyperlink{appendix_vol_treatment}{\beamergotobutton{Table}} 
    \item looking at first deal and repeated interaction clients separately 
     \\ \hyperlink{appendix_vol_first}{\beamergotobutton{Table}} 
   \end{wideitemize}}
  \only<5>{\Large \centering The bank \textcolor{red}{increases the volume of deals} with acquired clients \\ \vspace{.3cm} 
  The increase covers \textcolor{red}{both syndicated lending and bonds} 
     \\ \hfill \hyperlink{appendix_vol_dscan}{\beamergotobutton{Table}} }
      \end{column}%
\end{columns}
\end{frame}

\note[itemize]{
  \item Here you drop relationships that don't come over
  \item Treat all clients with whom you eventually make deals as rel\_acq from the first year after the switch
  \item Not so much in SEO sample - consistent with investment banking business being separate from lending
  \item Clearly, having FEs in there makes the comparison with the sample median not 100\% kosher
}

%%%%%%%%%%%%%%%%%%%%%%%%%%%%%%%%%%%%%%%
% CONCLUSION / EXTENSIONS / NEXT STEPS
%%%%%%%%%%%%%%%%%%%%%%%%%%%%%%%%%%%%%%%
\section{Conclusion}
\begin{frame}{Conclusion}
In sum, bankers appear to be an important piece in explaining the creation of lending relationships. They \textcolor{red}{facilitate the matching} between firms and banks. \\ \vspace{.5cm} 
\textcolor{red}{Open questions \& next steps:} 
\begin{wideitemize}
  \item \textcolor{red}{Identification} - Sources of exogenous variation in the probability of switching, e.g.,  restrictions in labor mobility or drop in bank performance/executive pay 
  \item Role of \textcolor{red}{bank culture} - Are bankers more likely to leave banks with a ``toxic culture''?
  \item Role of \textcolor{red}{demographics} - Are female bankers better at forming strong client relationships? Are they more or less likely to switch banks?
\end{wideitemize}
\end{frame}

\begin{transitionframe}
  \begin{center}
    \Huge Thank you!
  \end{center}
\end{transitionframe}


%%%%%%%%%%%%%%%%%%%%%%%%%%%%%%%%%%%%%%%%%%%%%%%%%%%%%%%%%%%%%%%%%%%%%%
%%%%%%%%%%%%%%%%%%%%%%%%%%%%%%%%%%%%%%%%%%%%%%%%%%%%%%%%%%%%%%%%%%%%%%
\appendix
\begin{transitionframe}
  \begin{center}
    \Huge Appendix
  \end{center}
\end{transitionframe}
%%%%%%%%%%%%%%%%%%%%%%%%%%%%%%%%%%%%%%%%%%%%%%%%%%%%%%%%%%%%%%%%%%%%%%
%%%%%%%%%%%%%%%%%%%%%%%%%%%%%%%%%%%%%%%%%%%%%%%%%%%%%%%%%%%%%%%%%%%%%%

\begin{frame}[label=appendix_bankers]{Data I - Individual bankers: Quality assurance}
Randomly sample 100 contracts to check quality of data:
\begin{wideitemize}
  \item 65\% of contracts feature signatures, other contracts are dropped
  \item 80\% of signatories are extracted successfully
\end{wideitemize}
\vspace{.3cm}
Talk to various bankers in commercial lending
\begin{wideitemize}
  \item Authorization of signature only for high ranking bankers
  \item Bankers that sign are the ones negotiating
  \item Titles are at the level of junior seniors
  \item LinkedIn search: Relationship bankers, commercial bankers     
\end{wideitemize}
\hfill \hyperlink{data_bankers}{\beamergotobutton{Back}}
\end{frame}

%%%%%%%%%%%%%%%%%%%%%%%%%%%%%%%%%%%%%%%%%%%%%%%%%%%%%%%%%%%%%%%%%%%
\begin{frame}[label=appendix_tenure]{Finding Ia - Tenure}
   \resizebox{.8\textwidth}{!}{ \centering 
\begin{tabular*}{\hsize}{@{\hskip\tabcolsep\extracolsep\fill}l*{6}{c}}
\toprule
                &\multicolumn{6}{c}{Pre-Switch Indicator (\%)}                                \\\cmidrule(lr){2-7}
                &\multicolumn{1}{c}{(1)}   &\multicolumn{1}{c}{(2)}   &\multicolumn{1}{c}{(3)}   &\multicolumn{1}{c}{(4)}   &\multicolumn{1}{c}{(5)}   &\multicolumn{1}{c}{(6)}   \\
\midrule
Tenure of banker (running)&     0.25** &     0.30***&            &            &            &            \\
                &   (2.57)   &   (3.07)   &            &            &            &            \\
 
Max tenure of banker&            &            &     0.50***&     0.55***&            &            \\
                &            &            &   (4.90)   &   (5.66)   &            &            \\
 
Tenure\(^{25\%-50\%}\)&            &            &            &            &     1.33***&     1.29***\\
                &            &            &            &            &   (3.39)   &   (3.44)   \\
 
Tenure\(^{50\%-75\%}\)&            &            &            &            &     2.37***&     2.31***\\
                &            &            &            &            &   (4.02)   &   (3.66)   \\
 
Tenure\(^{75\%-100\%}\)&            &            &            &            &     2.47***&     2.92***\\
                &            &            &            &            &   (3.05)   &   (3.54)   \\
\midrule
Observations    &   22,642   &   22,642   &    7,871   &    7,871   &   22,642   &   22,642   \\
R-squared       &     0.23   &     0.31   &     0.26   &     0.38   &     0.23   &     0.31   \\
\midrule Year FE &      Yes   &       No   &        Yes   &       No   &        Yes   &       No \\
Bank FE         &      Yes   &       No   &        Yes   &       No   &        Yes   &       No \\
Bank-Year FE    &      Yes   &       No   &        Yes   &       No   &        Yes   &       No \\
\bottomrule
\end{tabular*} }
\hfill \hyperlink{finding1}{\beamergotobutton{Back}}
\end{frame}

%%%%%%%%%%%%%%%%%%%%%%%%%%%%%%%%%%%%%%%%%%%%%%%%%%%%%%%%%%%%%%%%%%%
\begin{frame}[label=appendix_key]{Finding Ib - Key bankers}
   \resizebox{.8\textwidth}{!}{ \centering 
\begin{tabular*}{\hsize}{@{\hskip\tabcolsep\extracolsep\fill}l*{6}{c}}
\toprule
                &\multicolumn{2}{c}{All bankers}&\multicolumn{2}{c}{Non-key}&\multicolumn{2}{c}{Key}  \\\cmidrule(lr){2-3}\cmidrule(lr){4-5}\cmidrule(lr){6-7}
                &\multicolumn{1}{c}{(1)}   &\multicolumn{1}{c}{(2)}   &\multicolumn{1}{c}{(3)}   &\multicolumn{1}{c}{(4)}   &\multicolumn{1}{c}{(5)}   &\multicolumn{1}{c}{(6)}   \\
\midrule
\%Bank deals by banker&     0.18***&     0.29***&     0.39***&     0.69***&     0.13***&     0.17***\\
                &   (8.08)   &   (9.91)   &   (7.25)   &   (9.28)   &   (4.36)   &   (3.62)   \\
\midrule
Observations    &   43,233   &   43,233   &   36,730   &   36,427   &    6,300   &    4,954   \\
R-squared       &     0.24   &     0.30   &     0.18   &     0.22   &     0.57   &     0.71   \\
\midrule Year FE &      Yes   &       No   &      Yes   &       No   &      Yes   &       No   \\
Bank FE         &      Yes   &       No   &      Yes   &       No   &      Yes   &       No   \\
Bank-Year FE    &       No   &      Yes   &       No   &      Yes   &       No   &      Yes   \\
\bottomrule
\end{tabular*}}
\hfill \hyperlink{finding1}{\beamergotobutton{Back}}
\end{frame}

%%%%%%%%%%%%%%%%%%%%%%%%%%%%%%%%%%%%%%%%%%%%%%%%%%%%%%%%%%%%%%%%%%%%%%%%%%%
%%%%%%%%%%%%%%%%%%%%%%%%%%%%%%%%%%%%%%%%%%%%%%%%%%%%%%%%%%%%%%%%%%%%%%%%%%%
\begin{frame}[label=appendix_initiation]{Finding IIa - 1. Initiation strict}
   \resizebox{.8\textwidth}{!}{ \centering 
   \begin{tabular*}{\hsize}{@{\hskip\tabcolsep\extracolsep\fill}l*{6}{c}}
\toprule
Dep. variable: &\multicolumn{5}{c}{Initiation\_strict}                          \\\cmidrule(lr){2-6}
&\multicolumn{1}{c}{(1)}   &\multicolumn{1}{c}{(2)}   &\multicolumn{1}{c}{(3)}   &\multicolumn{1}{c}{(4)}   &\multicolumn{1}{c}{(5)}   \\
\midrule
Rel\_acq        &     0.06** &     0.07** &     0.12***&            &            \\
                &   (2.42)   &   (2.51)   &   (4.11)   &            &            \\
 
Rel\_acq\(^{5yr}\)&            &            &            &     0.11***&            \\
                &            &            &            &   (3.88)   &            \\
 
Rel\_acq\(^{abs}\)&            &            &            &            &     0.06***\\
                &            &            &            &            &   (3.32)   \\
\midrule
Observations    &  861,444   &  861,444   &  861,444   &  847,106   &  834,470   \\
R-squared       &     0.03   &     0.07   &     0.40   &     0.39   &     0.39   \\
\midrule Year FE &      Yes   &      Yes   &       No   &       No   &       No   \\
Firm FE         &      Yes   &       No   &       No   &       No   &       No   \\
Firm-Bank FE    &       No   &      Yes   &      Yes   &      Yes   &      Yes   \\
Bank-Year FE    &       No   &       No   &      Yes   &      Yes   &      Yes   \\
Firm-Year FE    &       No   &       No   &      Yes   &      Yes   &      Yes   \\
\bottomrule \end{tabular*}
}
\hfill \hyperlink{finding2_init}{\beamergotobutton{Back}}
\end{frame}

%%%%%%%%%%%%%%%%%%%%%%%%%%%%%%%%%%%%%%%%%%%%%%%%%%%%%%%%%%%%%%%%%%%%%%%%%%%
\begin{frame}[label=appendix_initiation_nofirst]{Finding IIa - 2. Ignoring first deal}
   \resizebox{.8\textwidth}{!}{ \centering 
   \begin{tabular*}{\hsize}{@{\hskip\tabcolsep\extracolsep\fill}l*{6}{c}}
\toprule
                Dep. variable: &\multicolumn{5}{c}{Initiation}                                  \\\cmidrule(lr){2-6}
                &\multicolumn{1}{c}{(1)}   &\multicolumn{1}{c}{(2)}   &\multicolumn{1}{c}{(3)}   &\multicolumn{1}{c}{(4)}   &\multicolumn{1}{c}{(5)}   \\
\midrule
Rel\_acq\_nofirst&     0.07** &     0.09** &     0.15***&            &            \\
                &   (2.35)   &   (2.35)   &   (3.88)   &            &            \\
 
Rel\_acq\_nofirst\(^{5yr}\)&            &            &            &     0.14***&            \\
                &            &            &            &   (3.61)   &            \\
 
Rel\_acq\_nofirst\(^{abs}\)&            &            &            &            &     0.09***\\
                &            &            &            &            &   (3.63)   \\
\midrule
Observations    &  858,844   &  858,844   &  858,844   &  844,504   &  834,668   \\
R-squared       &     0.03   &     0.08   &     0.42   &     0.41   &     0.41   \\
\midrule Year FE &      Yes   &      Yes   &       No   &       No   &       No   \\
Firm FE         &      Yes   &       No   &       No   &       No   &       No   \\
Firm-Bank FE    &       No   &      Yes   &      Yes   &      Yes   &      Yes   \\
Bank-Year FE    &       No   &       No   &      Yes   &      Yes   &      Yes   \\
Firm-Year FE    &       No   &       No   &      Yes   &      Yes   &      Yes   \\
\bottomrule \end{tabular*}
}
\hfill \hyperlink{finding2_init}{\beamergotobutton{Back}}
\end{frame}

%%%%%%%%%%%%%%%%%%%%%%%%%%%%%%%%%%%%%%%%%%%%%%%%%%%%%%%%%%%%%%%%%%%%%%%%%%%
\begin{frame}[label=appendix_initiation_treat]{Finding IIa - 3. Different treatments}
   \resizebox{.8\textwidth}{!}{ \centering 
   \begin{tabular*}{\hsize}{@{\hskip\tabcolsep\extracolsep\fill}l*{6}{c}}
\toprule
                  Dep. variable: &\multicolumn{5}{c}{Initiation}                                  \\\cmidrule(lr){2-6}
                &\multicolumn{1}{c}{(1)}   &\multicolumn{1}{c}{(2)}   &\multicolumn{1}{c}{(3)}   &\multicolumn{1}{c}{(4)}   &\multicolumn{1}{c}{(5)}   \\
\midrule
Rel\_acq        &     0.07** &     0.09** &     0.14***&            &            \\
                &   (2.37)   &   (2.38)   &   (3.80)   &            &            \\
 
Rel\_acq\(^{5yr}\)&            &            &            &     0.12***&            \\
                &            &            &            &   (3.56)   &            \\
 
Rel\_acq\(^{abs}\)&            &            &            &            &     0.07***\\
                &            &            &            &            &   (3.36)   \\
\midrule
Observations    &  861,444   &  861,444   &  861,444   &  847,106   &  834,470   \\
R-squared       &     0.03   &     0.08   &     0.42   &     0.41   &     0.41   \\
\midrule Year FE &      Yes   &      Yes   &       No   &       No   &       No   \\
Firm FE         &      Yes   &       No   &       No   &       No   &       No   \\
Firm-Bank FE    &       No   &      Yes   &      Yes   &      Yes   &      Yes   \\
Bank-Year FE    &       No   &       No   &      Yes   &      Yes   &      Yes   \\
Firm-Year FE    &       No   &       No   &      Yes   &      Yes   &      Yes   \\

\bottomrule \end{tabular*}
}
\hfill \hyperlink{finding2_init}{\beamergotobutton{Back}}
\end{frame}

%%%%%%%%%%%%%%%%%%%%%%%%%%%%%%%%%%%%%%%%%%%%%%%%%%%%%%%%%%%%%%%%%%%%%%%%%%%
%%%%%%%%%%%%%%%%%%%%%%%%%%%%%%%%%%%%%%%%%%%%%%%%%%%%%%%%%%%%%%%%%%%%%%%%%%%
\begin{frame}[label=appendix_vol_treatment]{Finding IIb - 1. Volume - Different treatment}
   \resizebox{.8\textwidth}{!}{ \centering 
 \begin{tabular*}{\hsize}{@{\hskip\tabcolsep\extracolsep\fill}l*{6}{c}}
\toprule
                &\multicolumn{6}{c}{Log Deal Volume}                                          \\\cmidrule(lr){2-7}
                &\multicolumn{1}{c}{(1)}   &\multicolumn{1}{c}{(2)}   &\multicolumn{1}{c}{(3)}   &\multicolumn{1}{c}{(4)}   &\multicolumn{1}{c}{(5)}   &\multicolumn{1}{c}{(6)}   \\
\midrule
Rel\_acq        &     0.65***&     0.72***&     0.61***&     0.30***&            &            \\
                &   (8.62)   &   (4.09)   &   (5.18)   &   (3.80)   &            &            \\
 
Rel\_acq\(^{5yr}\)&            &            &            &            &     0.70***&            \\
                &            &            &            &            &   (6.07)   &            \\
 
Rel\_acq\(^{abs}\)&            &            &            &            &            &     3.47***\\
                &            &            &            &            &            &   (6.58)   \\
\midrule
Observations    &  809,108   &  809,108   &  809,108   &  809,108   &  807,764   &  806,292   \\
R-squared       &     0.07   &     0.14   &     0.16   &     0.51   &     0.16   &     0.16   \\
\midrule Year FE &      Yes   &      Yes   &       Yes   &       Yes   &       Yes  &       Yes   \\
Firm FE         &      Yes   &       No   &       No   &       No   &       No&       No   \\
Firm-Bank FE    &       No   &      Yes   &      Yes   &      Yes   &      Yes   &      Yes   \\
Bank-Year FE    &       No   &       No   &      Yes   &      Yes   &      Yes   &      Yes   \\
\bottomrule
\end{tabular*}
}
\hfill \hyperlink{finding2_vol}{\beamergotobutton{Back}}
\end{frame}

%%%%%%%%%%%%%%%%%%%%%%%%%%%%%%%%%%%%%%%%%%%%%%%%%%%%%%%%%%%%%%%%%%%%%%%%%%%
\begin{frame}[label=appendix_vol_first]{Finding IIb - Volume 2. First vs. repeat deals}
   \resizebox{.8\textwidth}{!}{ \centering 
 \begin{tabular*}{\hsize}{@{\hskip\tabcolsep\extracolsep\fill}l*{6}{c}}
\toprule
                  Dep. variable: &\multicolumn{3}{c}{Volume - First deal}&\multicolumn{3}{c}{Volume - Repeat deals}\\\cmidrule(lr){2-4}\cmidrule(lr){5-7}
                &\multicolumn{1}{c}{(1)}   &\multicolumn{1}{c}{(2)}   &\multicolumn{1}{c}{(3)}   &\multicolumn{1}{c}{(4)}   &\multicolumn{1}{c}{(5)}   &\multicolumn{1}{c}{(6)}   \\
\midrule
Rel\_acq        &     0.81***&            &            &     1.29***&            &            \\
                &  (14.71)   &            &            &  (14.02)   &            &            \\
 
Rel\_acq\(^{5yr}\)&            &     0.96***&            &            &     1.16***&            \\
                &            &  (13.83)   &            &            &  (10.65)   &            \\
 
Rel\_acq\(^{abs}\)&            &            &     4.74***&            &            &     1.65***\\
                &            &            &  (11.62)   &            &            &   (6.43)   \\
\midrule
Observations    &  930,913   &  929,477   &  927,926   &  930,913   &  929,477   &  927,926   \\
R-squared       &     0.18   &     0.22   &     0.83   &     0.38   &     0.35   &     0.37   \\
\midrule Year FE &      Yes   &      Yes   &      Yes   &      Yes   &      Yes   &      Yes   \\
Firm FE         &       No   &       No   &       No   &       No   &       No   &       No   \\
Firm-Bank FE    &      Yes   &      Yes   &      Yes   &      Yes   &      Yes   &      Yes   \\
Bank-Year FE    &      Yes   &      Yes   &      Yes   &      Yes   &      Yes   &      Yes   \\
\bottomrule
\end{tabular*}
}
\hfill \hyperlink{finding2_vol}{\beamergotobutton{Back}}
\end{frame}


%%%%%%%%%%%%%%%%%%%%%%%%%%%%%%%%%%%%%%%%%%%%%%%%%%%%%%%%%%%%%%%%%%%%%%%%%%%
\begin{frame}[label=appendix_vol_dscan]{Finding IIb - Volume 3. Deal category}
   \resizebox{.8\textwidth}{!}{ \centering 
 \begin{tabular*}{\hsize}{@{\hskip\tabcolsep\extracolsep\fill}l*{6}{c}}
\toprule
                 Dep. variable: &\multicolumn{5}{c}{Log Deal Volume - Syndicated Loans}          \\\cmidrule(lr){2-6}
                &\multicolumn{1}{c}{(1)}   &\multicolumn{1}{c}{(2)}   &\multicolumn{1}{c}{(3)}   &\multicolumn{1}{c}{(4)}   &\multicolumn{1}{c}{(5)}   \\
\midrule
Rel\_acq        &     0.33***&     0.18   &     0.20** &            &            \\
                &   (4.62)   &   (1.40)   &   (2.04)   &            &            \\
 
Rel\_acq\(^{5yr}\)&            &            &            &     0.37***&            \\
                &            &            &            &   (3.06)   &            \\
 
Rel\_acq\(^{abs}\)&            &            &            &            &     1.84***\\
                &            &            &            &            &   (5.09)   \\
\midrule
Observations    &  574,769   &  574,769   &  574,769   &  574,032   &  573,293   \\
R-squared       &     0.08   &     0.15   &     0.17   &     0.17   &     0.17   \\
\midrule Year FE &      Yes   &      Yes   &      Yes   &      Yes   &      Yes   \\
Firm FE         &      Yes   &       No   &       No   &       No   &       No   \\
Firm-Bank FE    &       No   &      Yes   &      Yes   &      Yes   &      Yes   \\
Bank-Year FE    &       No   &       No   &      Yes   &      Yes   &      Yes   \\
\bottomrule
\end{tabular*}
}
\hfill \hyperlink{finding2_vol}{\beamergotobutton{Back}}
\end{frame}

\begin{frame}[label=appendix_vol_bond]{Finding IIb - Volume 3. Deal category}
   \resizebox{.8\textwidth}{!}{ \centering 
 \begin{tabular*}{\hsize}{@{\hskip\tabcolsep\extracolsep\fill}l*{6}{c}}
\toprule
                Dep. variable: &\multicolumn{5}{c}{Log Deal Volume - Bonds}                     \\\cmidrule(lr){2-6}
                &\multicolumn{1}{c}{(1)}   &\multicolumn{1}{c}{(2)}   &\multicolumn{1}{c}{(3)}   &\multicolumn{1}{c}{(4)}   &\multicolumn{1}{c}{(5)}   \\
\midrule
Rel\_acq        &     0.45***&     0.51***&     0.43***&            &            \\
                &   (5.31)   &   (5.02)   &   (6.66)   &            &            \\
 
Rel\_acq\(^{5yr}\)&            &            &            &     0.45***&            \\
                &            &            &            &   (6.79)   &            \\
 
Rel\_acq\(^{abs}\)&            &            &            &            &     2.52***\\
                &            &            &            &            &   (4.75)   \\
\midrule
Observations    &  288,896   &  288,896   &  288,896   &  287,820   &  286,598   \\
R-squared       &     0.14   &     0.19   &     0.21   &     0.21   &     0.20   \\
\midrule Year FE &      Yes   &      Yes   &      Yes   &      Yes   &      Yes   \\
Firm FE         &      Yes   &       No   &       No   &       No   &       No   \\
Firm-Bank FE    &       No   &      Yes   &      Yes   &      Yes   &      Yes   \\
Bank-Year FE    &       No   &       No   &      Yes   &      Yes   &      Yes   \\
\bottomrule
\end{tabular*}
}
\hfill \hyperlink{finding2_vol}{\beamergotobutton{Back}}
\end{frame}

\begin{frame}[label=appendix_vol_seo]{Finding IIb - Volume 3. Deal category}
   \resizebox{.8\textwidth}{!}{ \centering 
 \begin{tabular*}{\hsize}{@{\hskip\tabcolsep\extracolsep\fill}l*{6}{c}}
\toprule
                Dep. variable: &\multicolumn{5}{c}{Log Deal Volume - SEOs}                      \\\cmidrule(lr){2-6}
                &\multicolumn{1}{c}{(1)}   &\multicolumn{1}{c}{(2)}   &\multicolumn{1}{c}{(3)}   &\multicolumn{1}{c}{(4)}   &\multicolumn{1}{c}{(5)}   \\
\midrule
Rel\_acq        &     0.12   &     0.09   &     0.02   &            &            \\
                &   (0.94)   &   (0.30)   &   (0.06)   &            &            \\
 
Rel\_acq\(^{5yr}\)&            &            &            &     0.11   &            \\
                &            &            &            &   (0.46)   &            \\
 
Rel\_acq\(^{abs}\)&            &            &            &            &     0.61   \\
                &            &            &            &            &   (1.43)   \\
\midrule
Observations    &  201,741   &  201,741   &  201,741   &  201,330   &  200,873   \\
R-squared       &     0.06   &     0.09   &     0.10   &     0.10   &     0.10   \\
\midrule Year FE &      Yes   &      Yes   &       No   &      Yes   &      Yes   \\
Firm FE         &      Yes   &       No   &       No   &       No   &       No   \\
Firm-Bank FE    &       No   &      Yes   &      Yes   &      Yes   &      Yes   \\
Bank-Year FE    &       No   &       No   &      Yes   &      Yes   &      Yes   \\

\bottomrule
\end{tabular*}
}
\hfill \hyperlink{finding2_vol}{\beamergotobutton{Back}}
\end{frame}
% Findings with different rel\_acq definitions
% Findings with initiation interaction 

\end{document}
