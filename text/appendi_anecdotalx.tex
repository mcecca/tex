

This section provides anecdotal evidence of both bankers being actively poached by competing lenders, and their ability to move their clients with them.  


\begin{enumerate}
	\item The following article details how JPMorgan poached commercial bankers from various competitors to bolster its lending business:\footnote{Article available at \url{https://www.reuters.com/article/us-jp-morgan-europe/jpmorgan-hires-commercial-bankers-leaders-across-europe-asia-idUSKCN1QG2SQ}} \\
		
	
	\textit{\textbf{JPMorgan Chase \& Co on Wednesday named half a dozen people to a commercial banking team in Europe and new international and Asia-Pacific regional leaders, as the U.S. bank closes in on business clients it hopes to poach from rivals abroad.}} %For two years, JPMorgan’s commercial banking business has been building a list of around 1,500 middle-market European companies that it wants to attract through its global approach to investment banking, credit, hedging and treasury services.}
\vspace{1em}

	\item Smaller lenders follow the same strategy:\footnote{Article available at \url{https://www.boyden.com/media/small-regional-banks-find-ease-poaching-large-bank-talent-588520/index.html}}\\
		
		\textit{As CEO of Manhattan-based Signature Bank (SBNY), DePaolo recently set out to recruit four teams of veteran bankers from large rivals [...] The strategy gives smaller firms a crack at picking off large rivals' clients, says Jeff Davis, senior analyst at FTN Financial, a unit of First Horizon National Corp. (FHN). That's because \textbf{business customers are notoriously loyal to their bankers.} Signature, which has 22 offices in the New York metro area, has virtually raided the ranks of what was once North Fork Bank, a former 350-branch Long Island lender. More than 80 North Fork alumni have moved to Signature since Capital One Financial Corp.} %(COF) agreed to purchase North Fork last year.
		
		
		\textit{Jim Schmitz, president of middle Tennessee for Regions, based in Birmingham, intends to add more Nashville bankers at the beginning of 2013. \textbf{He said he would try to recruit from rival banks if he can find local lenders who can bring ``relationships with companies we don't already have. The competition for people and customers is as a fierce as I've ever seen it},'' he said. }
\vspace{1em}

	\item Banks strategically time their poaching of talent from competitors, for example after a merger:\footnote{Article available at \url{https://www.americanbanker.com/news/citizens-looks-to-poach-bb-t-suntrust-talent-but-should-expect-a-fight}}\\
		
		\textit{The \$161 billion-asset Citizens has already recruited one team of commercial bankers away from SunTrust, and its top commercial banker said Tuesday that he expects the merger — the industry’s largest in more than a decade — to lead to more banker defections. \textbf{McCree said Tuesday that Citizens has been adding around 300 new clients every year across its footprint, and it’s done so primarily by hiring local bankers in its newer markets. [...] ``They are bringing clients with them, which is one of my goals when I hire people'', he said.} %McCree is Vice Chairman and Head of Commercial Banking at Citizens} 
	[...] Several banks in the Southeast and mid-Atlantic have said that they intend to go after customers and bankers that might be looking to leave BB\&T-SunTrust when that merger closes. }
\vspace{1em}

%additional similar info:
	\item Bankers are aware of their important role in forming relationships to borrowers:\footnote{Article available at \url{https://www.spglobal.com/marketintelligence/en/news-insights/research/ma-creates-poaching-opportunities-for-commercial-credits}}\\
		
		\textit{\textbf{Bankers often talk about C\&I lending as a relationship-driven business}, so an acquisition could force large clients to reconsider their lending partners. Commercial borrowers often select banks for reasons beyond the loan's terms and pricing, valuing institutions that can also deliver cash management, debt syndication or other services.}
\vspace{1em}

	\item Banks are aware of the risk of defecting bankers taking clients with them, and strictly enforce cool down periods and non compete agreements:\footnote{Article available at 	\url{https://www.finews.com/news/english-news/34276-marco-illy-ubs-credit-suisse-investment-bank-switzerland-credit-suisse-notice-contract-dissolved-andrea-orcel-axel-lehmann-christine-novakovic}}\\
	
		\textit{The prolific rainmaker received an \textbf{immediate termination from Credit Suisse after the bank accused him of breaking rules governing contact with clients}. The basis of trust between Illy and UBS eroded as a result of the episode, which represents a rare public glimpse into how competitive banks are with their talent.}
\vspace{1em}

	\item The following excerpt describes how borrowers moved their banking relationship as bankers switched to a new lender:\footnote{Article available at 	\url{https://www.chicagotribune.com/news/ct-xpm-2008-02-18-0802170180-story.html}}\\

		\textit{[CEO Dan Ariens] approached his lender, LaSalle Bank, to see if it would ramp up his \$30 million credit line to \$45 million. But LaSalle, which on Oct. 1 was bought by Bank of America Corp., would never get the additional business. [....] \textbf{Ariens, in fact, ended up moving nearly all of his banking relationships to PrivateBancorp Inc. The Chicago-based bank hired 56 managing directors in the fourth quarter, most of them from LaSalle}, and posted a 12 percent increase in loans compared with the year-ago quarter. \textbf{``It felt natural to stay with the people we knew,''} he said.}
\vspace{1em}

%another article not sure it adds much:
%https://www.efinancialcareers.com/news/2017/02/banks-want-to-hire-banking-advisory-veterans-and-ai-pros

\end{enumerate}

\clearpage

