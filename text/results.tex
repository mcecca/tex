This section we first investigate which characteristics of a banker's portfolio correlate with a higher probability of switching banks. Second, we take the perspective of the bank that acquires client relationships of the banker: Do the banks benefit from the banker's switch? Finally, we explore what types of deals are driving these results. 

\subsection{Which bankers change employer?}

If individual bankers are the key to maintaining borrower relationships for a bank's existing clients, poaching bankers from another lender can be a fast way for a bank to obtain new lending relationships. Appendix \ref{app:anecdotal} provides large amounts of anecdotal evidence that supports this conjecture. %supports this idea, e.g., in the prominent example of Iqbal Khan whose old employer hired private investigator to make sure that he does not ``steal business leads'' \citep{Fortune.2019}. 
We posit that the composition and value of a banker's client portfolio is an important driver of the likelihood of getting poached, and switching employer. 

To examine the importance of the client portfolio in our sample, we estimate specification~\ref{eq:switch}. The independent variable is $Banker~hired_{i,t}$, an indicator variable for the year $t$ when banker $i$ starts working at a new bank, on $\#Clients^{\theta}_{i,t-1}$, various measures of the number of clients the banker had in her portfolio in the previous year. %These include the total number of clients, the number of small and large clients, and the number of clients for which banker $i$ is the single or multiple contact respectively. 
Even columns include bank ($\gamma_{j}$) and year fixed effects ($\delta_{t}$) to control for time invariant bank characteristics and general time trends. Odd columns present specification with bank$\times$time fixed effects to account for the possibility that a bank might want to expand in a given year and is therefore hiring bankers. Intuitively, these specifications therefore compare the likelihood that a certain bank hires one of two bankers at the same point in time, and asks if each banker's client portfolio composition impacts their likelihood of getting hired. Standard errors are clustered two-dimensionally at the bank and banker level to account for arbitrary correlations in error terms within banks across time, and within years across bankers.

\begin{equation}
Banker~hired_{i,t} = \beta_1 \#Clients^{\theta}_{i,t-1} + \beta_2 \gamma_{j} +\beta_3 \delta_t + \epsilon_{i,t} \label{eq:switch}
\end{equation}

Panel A of Table~\ref{tab:banker_clientno} reports the results of this regression. The first column shows that having a personal relationship with one additional client correlates with an increase in the switching probability by 0.2\%. %While small, this point estimate is economically meaningful, corresponding to 13\% of the unconditional switching probability in each year. A one standard deviation increase in the number of clients (4.29) translates in an increase in switching probability of 0.9\%. 
When we introduce bank$\times$year fixed effects in column two, the coefficient becomes 0.15\%. This effect is economically large. A one standard deviation increase in a banker's number of portfolio firms (4.29 clients)  corresponds to a roughly 50\% increase in likelihood of getting hired by a bank, compared to the unconditional probability. 

\begin{center} - Table~\ref{tab:banker_clientno} - \end{center}

In columns (3) and (4), we investigate whether it is not just the number, but also the size of clients in a banker's portfolio that matter for their propensity to get poached by competing lenders. There is no clear theoretical prediction if smaller clients matter more or less than larger ones. On the one hand, relationship lending is most important for small firms \citep{Petersen.1994c}, which means bankers might be more successful in moving their small clients to a new bank. 
%We posit that smaller clients are more likely to have stronger personal relationships with their bankers as 
Therefore, having more small clients in ones' portfolio could increase the likelihood of the borrowers following the banker to the new bank. %\footnote{It could also be the case that the soft information is collected by the bank as opposed to the banker \citep{Rajan.1992}, making small firms \emph{less} likely to switch.} 
On the other hand, a banker with many larger clients might be more attractive to new banks, as larger clients potentially generate more, and more lucrative deals. Which of these channels dominates is ultimately an empirical question, and we test these hypothesis by separately estimating the effect of the number of small and large clients in columns 3 and 4. For the purpose of this analysis, we define clients are those with assets below the 25-th percentile in our sample. 

The results in column 3 suggest that both the number of small and large clients impact the likelihood of a bankers witching banks. The coefficient estimate for the number of small clients is nominally larger than that for large clients, at 0.63 as opposed to 0.25 for large clients. However, since there are significantly fewer small borrowers in our sample, a one standard deviation increase in the number of large clients increases the likelihood of switching by 1.1\%, about five times as much as a one standard deviation increase in the number of small clients. %The same increase in the number of large clients correlates with a probability that is 1.1\% higher. 
Once we add bank$\times$year fixed effects in column 4, the coefficient on small clients drops to 0.34 and is no longer statistically significant, while that on large clients actually increases to 0.17, and remains statistically highly significant. In sum, these results indicate that a portfolio of larger clients makes bankers particularly attractive targets for other lenders in the labor market.

\textcolor{red}{CH: A referee will likely push us on the definition of "small" here, at 25\% of assets (and 3\% of the sample). Any chance these results go through with a 50\% cutoff, or maybe even better, in continuous form where we take the avg size of clients inside the portfolio?}
%The effect of large clients on the other hand, remains largely unchanged. In sum it appears that having large clients in ones portfolio is particularly important in determining the probability of switching to a new bank. 

In columns (5) and (6) we distinguish between clients that have a relationship with multiple bankers and those for which the switching banker is their single contact at the bank. 

\textcolor{blue}{Firms with multiple banking relationships are less likely to switch banks \citep{Ioannidou.2010b}. and are more shielded against bank-specific risks \citep{Degryse.2011}. Hence, we assume that such firms are also less likely to follow the banker to a new bank.} \textcolor{red}{It is cool to cite the literature, but I think the connection is not 100\% there. These papers are about multiple bank relationships, our hypothesis is about multiple banking relationships. I re-wrote. (I understand the multi contact variableto mean "multiple bankers within a bank", consistent with page 7)}

Some firms interact with more than one banker within their lender. The literature on institutional relationships between banks and borrowers has found that spreading lending relationships, and hence information, between multiple lenders reduces the importance and bargaining power of each individual lender. In our context, we hypothesize analogously that a banker has less sway over a borrowers if  the borrower has multiple points of contact within the bank. If this is the case, then we would expect that relationships to borrowers for which a banker is the sole point of contact are particularly valuable. 
%On the other hand, single contact firms might be more dependent on the relationship with the switching banker. 

Our results in columns 5 and 6 of Table \ref{tab:banker_clientno} are consistent with this hypothesis. A one standard deviation increase in the number of single contact clients correlates with a probability of getting hired each year that is about 1.1\% higher, almost double the unconditional likelihood. In contrast, the coefficient estimates for the number of clients clients with multiple banker relationships are economically and statistically insignificant in both specifications. %When we introduce bank$\times$year fixed effects, the coefficient becomes even negative, while the coefficient of single contact clients remains positive and significant. 

We provide a number of additional tests that support these results in Appendix \ref{app:tables}. In Appendix Table~\ref{tab:banker_dealno}, we perform the same tests but using the number of deals instead of the number of clients. We do this in an attempt to account for the volume of business that the banker generates.%\footnote{Ideally, we would want to know the fees that the bank obtains due to the personal relationships of the banker. Unfortunately, this information is not available.} 
All results from of these regressions are consistent with the previous findings both in terms of economic magnitude and statistical significance.
%: A one standard deviation in the number of deals that the banker signs increases her switching probability by 0.6\%. After accounting for bank$\times$year fixed effect, this becomes 0.4\%. The number of deals with small clients is positive and significant only in the less stringent specification. The number of deals with large clients however is positive and significant: A one standard deviation increase in these types of deals translates to a switching probability that is 0.7\% higher. Finally, only deals with single contact clients positively influence the switching probability. A one standard deviation increase in these types of deals correlates to a 0.3\% higher probability of switching.

Taken together these results suggest that having personal relationships with more clients is beneficial for the probability of changing employer. Particularly important are large clients and those for whom the switching banker is the single contact to the bank. 

\subsection{Banks benefit from poaching banker from competitors} \label{sec:initiation}

%In the previous section we have shown that the size of the client portfolio is an important characteristics in explaining the likelihood of a bankers switching employer. 
We now investigate the perspective of the bank and ask if hiring new bankers from competitors actually benefit the lender. We start by showing that hiring a banker with a personal relationship to a specific firm indeed increases the likelihood to initiate new business relationships with this clients. %Second, we show that the deal volume closed with these clients is higher relative to the existing client base of the banks. This effect can be mostly attributed to syndicated lending and bond underwriting. Finally, our results are not driven by a one-time boost in deal activity, but appear to be long lasting.

\subsubsection{Banks expand their client portfolio after a new banker joins}
For this analysis, we create a balanced panel of all bank-firms in our sample, as in Figure \ref{fig:ex_bank}. %and complement this with the employment history of the bankers. 
When we observe a banker switching to a new bank, we treat all firms with whom the banker had contact at the old bank as personal relationships that the bank ``acquires''. We set the relationship acquired variable to one for all these bank-firm pairs in the years after the switch. We then estimate regression~\ref{eq:init}, where the dependent variable is $Inititaiton_{i,j,t}$, an indicator that takes the value of one if bank $i$ has the first deal with firm $j$ in year $t$, or if the last deal with firm $j$ was more than 5 years ago.  We also define $Inititaiton\_strict_{i,j,t}$ that identifies only the first deal with firm $j$. A deal can be either a syndicated loan, a bond, a SEO offering, or an M\&A advisory contract.

\begin{equation} \label{eq:init}
Initiation_{i,j,t} = \beta_1 Rel\_acq^\theta_{i,j,t} + \beta_2 \gamma_i + \beta_3 \phi_j + \beta_4 \delta_{t} + \epsilon_{i,j,t}
\end{equation}

The explanatory variable is the treatment identifier $Rel\_acq^\theta_{i,j,t}$ that identifies all firms $j$ for which bank $i$ has acquired a personal relationship at time $t$. $\theta$ differentiates between the various types of treatment, ever treated ($Rel\_acq$), treated within the last five years ($Rel\_acq^{5yr}$), and absorptive treatment ($Rel\_acq^{abs}$), i.e., treated only during the year of the switch. We further saturate the model with various types of firm, bank, and time fixed effects ()$\gamma_{i}$, $\phi_{j}$, and $\delta_{t}$). 
Standard errors are clustered two-dimensionally at the firm and lender level. Table~\ref{tab:init} reports the regression results. \textcolor{red}{CH: since we use the "treatment" language, we might be forced to show parallel trends (that's a common ref response to reading the word "treatment")}

\begin{center} - Table~\ref{tab:init} - \end{center}

The coefficient estimate on $Rel\_acq$ in column (1) shows that acquiring a personal relationship increases the initiation likelihood by \textcolor{red}{7\% CH: The summary statistics report initiation as percent - is it an indicator here?) if so, should report it as indicator in sum stat I think?} per year. This increase represents  an economically sizable effect, more than twice the unconditional mean of initiation of 5.2\%. \textcolor{red}{CH: I am not sure if we can do the "a one sd increase in rel acquired leads to... calculation here. The sd is 16, the coefficient is 7, which  implies a close to 100\% chance of getting the client. But I am also not sure about how to think of this back of envelope calc in the presence of all our FE  }
This specification includes firm fixed effects, which absorb all time invariant firm level heterogeneity, such as certain firms being more likely to raise capital from new banks and having more relationships to bankers. Year fixed effects control for time varying effects, such as periods with stronger economic growth being associated with more switching of bankers as well as borrowers. Finally bank fixed effects control for time invariant heterogeneity in the propensity of banks to acquire both new clients and new bankers. 
Moreover, the data structure is very conservative. We code \emph{all} firms with whom a banker had deals in the past - regardless of when - as acquired relationships in \emph{all} years after the banker joins.

One potential alternative explanation to this finding could be that there are unobservable, time invariant characteristics that make the match between a specific firm and bank a particularly good match, while also making the bank an attractive employer for bankers. For example, a bank might specialize in a certain industry or a certain subset of the lending market that both make it attractive to bankers and borrowers in that industry . 
%firms change banks not because of their bankers switching, but because of particular characteristics of the new bank. 
In column (2), we address this issue by introducing a firm$\times$bank fixed effect. Economically, this means that we compare the same firm-bank pair over time, before and after a banker with a prior relationship to the firm switches to the bank. These fixed effects also absorb all time invariant characteristics, such as size, geographic concentration, or culture, on both the firm level and the bank level. This includes pair specific characteristics, for example the bank being specialized in the firm's industry. The coefficient of interest remains positive and significant in this specification.

One concern could be that a bank expanding its operations is both hiring more bankers, and acquiring new firms. In column (3) we therefore include bank$\times$year fixed effects that capture the average propensity of banks to acquire new firms (and new bankers). The variation we exploit in this specification is therefore comparing firms that had a personal relationship to a banker acquired by the bank in a given year to those firms for which no banker was acquired. Intuitively, these fixed effects therefore control for loan \textit{supply} conditions. 

Finally, we also include firm$\times$year fixed effects. These absorb any time varying firm characteristics, such as a firm seeking loans from new lenders to finance an acquisition. Importantly, these firm-year fixed effects also nest industry year fixed effects that control for confounding factors such as strong growth within a certain industry. Effectively, we are now comparing the propensity of a certain firm to initiate a new lending in a given year, choosing between a lender that recently acquired a relationship banker and a lender that has not. Intuitively, these fixed effects control for loan \textit{demand} conditions \citep{jakovljevic2020}. Even with this tight econometric specification, the effect of acquiring a personal relationship to a firm on the probability of initiation is positive, economically meaningful, and highly statistically significant: A one standard deviation increase in $Rel\_acq$ (.17) correlates to a 2\% increase in initiation probability (10\% of a standard deviation). \textcolor{red}{CH: I now better understand these numbers. I think we can improve readability and clarity a lot by making numbers identical, i.e. always report \% or always report digits. Personally,, I would like to have all indicators as percentages bc most numbers and coefficients are very small, sow ould improve presentation a lot. What do you think?}

In columns (4) and (5) we account for the fact that the relationships that a bank acquires decay over time. Specifically, we attribute initiations to acquired relationships only within a time-frame of 5 years and 1 year respectively. Our results remain virtually unchanged even when using the most conservative treatment definition: Acquiring a new client relationship makes initiation 12\% more likely in the 5-year period after the banker switches and 7\% more likely in the year of the switch.

It Appendix Table~\ref{tab:init_strict} we restrict the definition of the dependent variable such that it includes only relationships with new clients. Our results remain virtually unchanged, regardless of the treatment definition. Moreover, since we identify bankers changing banks using deals that they sign at the new employer, one concern could be that there is a mechanical component to our estimates. \textcolor{red}{CH: I rewrote the sentence before - compare it to the last version. This was a very painful lesson I had to learn over the last few years in how to correctly phrase and "hedge" these limitations} In Appendix Table~\ref{tab:init_cross} we replicate all our previous results whilst dropping the first deal that the bankers sign when they switch banks. Our results remain robust even in this specification. In sum, it seems that banks expand their client portfolio with firms for which they acquire personal relationships.

Not all borrowers rely on personal relationships to the same degree. Smaller, more opaque borrowers of worse credit quality often rely more heavily on relationship lending compared to other borrowers \textcolor{red}{CITE}. We therefore hypothesize that those borrowers are more likely to follow their bankers to new banks. We formally test this hypothesis in Table \ref{tab:init_cross}.

\begin{center} - Table~\ref{tab:init_cross} - \end{center}

Specifically, we interact our measure of a bank having acquired a banker with a personal relationship to a borrower with one of three proxies for the borrower's opacity. First, in column (1), an indicator if the borrower is rated as ``junk'', i.e. non investment grade. Second, in column (2, an indicator for whether the borrower's intangibles-to-asset- ratio is above the median, and finally, in column (3), we interact $rel_acq$ with an indicator for small firms with assets below the median. We find that, in all three cases, the main finding remains robust that a lender that acquires a personal relationship to a borrower by hiring a connected banker is significantly more likely to initiate a business relationship with the borrower. In addition, we find that the interactions of all three proxies for opacity load positive and statistically significantly. The economic magnitude of these estimates is substantial and represents an increase of between roughly 50\% and 100\% compared to the baseline estimate. Borrowers that rely more heavily on relationship lending therefore are more likely to follow their bankers to new lenders. 

\subsubsection{Banks increase deal volume to clients acquired through bankers}

%In the previous section we showed that when a bank acquires a personal relationship to a firm, the likelihood that the bank initiates a business relationship with that firm increases. 
Acquiring bankers and their portfolio of relationships is not an end in itself. Ultimately, banks aim to sell services to these borroers. In this section, we investigate if acquiring personal relationships also lead to an increase in the volume of deals that the bank closes? 

\begin{equation} \label{eq:vol}
Log\ Deal\ Volume_{i,j,t} = \beta_1 Rel\_acq^\theta_{i,j,t} + \beta_2 \gamma_i + \beta_3 \phi_j + \beta_4 \delta_{t} + \epsilon_{i,j,t}
\end{equation}

To answer this question we run regression~\ref{eq:vol}, where the dependent variable is the logarithm of the total deal volume that the bank $i$ signs with firm $j$ during year $t$. The main explanatory variable, $Rel\_acq^\theta_{i,j,t}$, is the indicator for the firms $j$ where bank $i$ has acquired a personal relationship as of time $t$.\footnote{For this part of the analysis we drop all firm-bank pairs for which a bank acquires a relationship but never closes a deal with. We do this since we are interested in the deal volume generated by the banker's relationships relative to the \emph{existing} deals. \textcolor{red}{We need to thinking a little more about this. So we limit the sample to either a) firms for which there were deals but never relationships and b) relationships and deals. We exclude relationships without deals? I think we had thought this through before but reading it with a cold eye it feels odd. In particular, why is this an issue for deal size but not for initiation? In general, we probably need to have a small appendix where we explain our sample (we definitely thought a lot about this a year ago when we started but it's not in the paper right now I think?})} %As in the previous section, we saturate the base model with bank ($\gamma_i$), firm ($\phi_j$), and time ($\delta_t$) fixed effects. In the most conservative specification we introduce fixed effects for every firm-bank, bank-year, and firm-year pair. Standard errors are clustered two-dimensionally at the firm and bank level.  

\begin{center} - Table~\ref{tab:main_dealsize} - \end{center}

Panel A of Table~\ref{tab:main_dealsize} shows the results of estimating regression~\ref{eq:vol}. We find a positive and significant effect of acquiring personal client relationships on the total deal volume with these clients. The point estimate in column (1) implies that a one standard deviation increase in $Rel\_acq$ (0.06) correlates to an increase in deal volume of ca. 4\% each year. From column (2) we can establish that accounting for time-invariant characteristics of the bank-firm pair (e.g. cultural proximity or industry specialization of the bank) does not change our results. Columns (3) and (4) show that the same holds when we account for loan supply and demand characteristics respectively. Columns (5) and (6) confirm our findings for a more conservative definition of the treatment identifier. Finally, Appendix Table~\ref{tab:deal_vol_nofirst} shows that all of these findings are robust to dropping the first deal that the bankers close at the new bank. 

In Panel B of Table~\ref{tab:main_dealsize} we document the interaction effect of initiating a new client relationship whilst acquiring a personal relationship to that client. This shows that the increase in deal volume we observe actually comes form clients with whom the bank had no business in the past. The comparison group in this specification will be all deals that the bank signs with existing customers for which no relationship has been acquired. Compared to these firms, a one standard deviation increase in $Initiation$ (0.23) translates in a boost of 0.74 standard deviations in deal volume with new clients (with whom \emph{no} personal relationship exists).

The coefficient of $Rel\_acq$ captures the business that a bank generates with \emph{existing} clients for which an additional personal relationship was acquired. The negative coefficient means that, if anything, acquiring such relationship is not helpful to generate additional business. 

The interaction term $Rel\_acq \times Initiation$ measures the boost in deal volume attributable to new clients for which a personal relationship has been acquired. The total effect corresponds to 28\% $[=(1.64-0.29)/4.91]$ of the effect of initiating a new client relationship. Hence, when a bank manages to win over a firm for which it acquired a personal relationship in the past it will generate almost a third more deal volume compared to the scenario when a bank brings over a firm without having a personal relationship first.

The results in columns (2) to (4) show that the results are robust to introducing controls for time-invariant bank-firm-pair characteristics, as well as for loan demand and supply factors. Models (5) and (6) show that restricting the treatment window to five years or even to one year does not have a meaningful effect on our results.\footnote{In the last column the coefficient of $Rel\_acq^{abs}$ becomes positive and significant. This is due to us treating all years after the relationship has been acquired as missing. In doing so we do not ``penalize'' the bank for the years after the relationship has been acquired and there were no deals.} Appendix Table~\ref{tab:deal_vol_int_strict} confirms that our results are robust to using a more conservative definition of $Initiation$ that identifies only new clients. In unreported analyses we can also confirm that the effects are even stronger if we restrict the sample to firm-bank-years where at least a deal is closed. We can thus ensure that the findings are not driven by existing clients not taking out loans. 

These results show that banks do not only expand their client portfolios after they acquire personal relationships to firms, but also increase the volume of deals that they close with these clients. This is particularly the case for \emph{new} clients for which a relationship has been acquired in the past.  

\textcolor{red}{OK, so I think this section is our weakest one right now. The data selection will raise flags and interpretation of coefficients is not easy. Let's think about this. Maybe we keep only panel B and move panel A into the appendix? Or we kick it all to the Appendix? It feels as if it really does not provide a lot of new info over the initiation results, and raises flags that could contaminate the rest of the paper?  I now think I remember that the key problem is that initiation takes value 1 for all years, but deal volume fluctuates (naturally). So we really penalize ourselves bc each "zero" counts over and over again, and even for a "one", i.e. where deals happen, we have 4 zero observations for each 1. I think  we need to present these results in the pre- post format as Table A5. So split A5, put the deal volume in here, keep the initiation pre- post as robustness A5. Than move the "deal cathegory" regressions (current table 6) to pre-post format. Also, we should make sure we always have the same number of obs in regressions (we lose some due to fixed effects.) the way to do that is to run once the most restrictive regression (with all FE) before the actual table, and the use the command "keep if e(sample) == 1". this limits the sample in all regressions to the most restrictive one, it's just a bit more clen bc we have no changes in sample across specs. }

%%%%%%%%%%%%%%%%%%%%%%%%%%%%%%%%%%%%%%%%%%%%%%%%%%%%%%%%
%% TOTAL DEALS BY CATEGORY
\subsubsection{Do banks cross-sell other products?}

\textcolor{red}{Re-do with pre/post. Decide if we want to keep presenting SEO if results remain insignificant. Consolidate into single table with two columns each?}

In Table~\ref{tab:main_dealsize_categ} we investigate if the relationships acquired through bankers allow banks to cross- sell products other than commercial loans. We separately estimate specification \ref{eq:vol} for the samples of syndicated loans, bonds, and SEOs. For each sub-sample we keep only the observations where either no deal occurs or where at least one syndicated loan, one bond, or one SEO was closed respectively. The dependent variable in Panel A is the logarithm of the total syndicated loans taken out by a firm. In Panels B it is the logarithm of the total bonds and in Panel C that of the total SEOs the bank underwrites.\footnote{For brevity we do not report the results for M\&A deals since the sample is quite small and the coefficients are insignificant. This is because (a) we require  the deal volume to be public and (b) the sample overlap is small.}  

\begin{center} - Table~\ref{tab:main_dealsize_categ} - \end{center}

Column (1) of Panel A and B shows that a one standard deviation increase in $Rel\_acq$ is associated with a respective increase in 1.3\% and 3.5\% in syndicated loans and bond underwriting respectively. While small, the point estimate roughly corresponds (for the average deal) to an additional USD 10 million and USD 33 million in syndicated loans and bonds that a bank closes with a firm during a single year. 

The coefficients in Panel A remain positive but become insignificant after we introduce firm $\times$ bank and bank $\times$ year fixed effects. However, after we account for loan demand in column (4), the effect is again significant. This stresses the importance of comparing the borrowing decision of a firm that chooses between two banks, one where it knows the banker from previous interactions, and one to which it has no connection. The last two columns highlight that our results are robust to using a stricter treatment identifier. In Panel B, all coefficients are positive and highly significant. This highlights the fact that our results are not driven exclusively by the syndicated loan market but extend to bond underwriting as well. 

In contrast, from Panel C we gather that the underwriting of Seasoned Equity Offerings is not at all influenced by personal relationships. This finding is in line with the idea of ``Chinese walls'' between the commercial lending and investment banking departments.\footnote{Clearly, our results do not exclude the possibility that the SEO and M\&A business is influenced by relationships with \emph{investment} bankers. We are arguing that bankers that work in the syndicated lending division of banks have no influence in facilitating the former type of deals.} 

\subsubsection{Are relationships long-lasting?}

We now turn to the question if hiring these bankers allows banks to establish long-lasting relationships to borrowers, or if hiring bankers with a portfolio of clients leads to one-off deals.

\begin{center} - Table~\ref{tab:first_repeat} - \end{center}

In Table~\ref{tab:first_repeat} we therefore examine whether the increase in deal volume is due to repeat business with the new clients or if it is a one-time boost. To answer this question we split separately examine the deal volume that a bank has with its clients into the first deal with a firm ($First~Deal$), and the deal volume attributable to subsequent deals ($Repeat~Deals$). If the boost in observed deal volume where to be due to a one-time interaction, we would find a significant effect only in the case of first deals, and not in subsequent years. 

Comparing columns (1) and (4) highlights that this is not the case. Coefficient estimates for all measures of relationships are statistically and economically significant for both first- and subsequent deals. If anything, the banks profit mostly from repeated interactions with the clients for whom they acquire personal relationships.\footnote{For brevity we report only the results from our most stringent econometric specification.} This holds even if we restrict the treatment identifier to the first 5 years after a personal relationship has been acquired as shown in columns (2) and (5). Only when we solely use the first deal that the banks sign with personal relationship clients is the first-deal-effect larger than the repeat-deal-effect. Therefore, it seems that when banks acquire personal relationships to firms they experience a long-lasting boost in deal volume with these companies. \textcolor{red}{Let's think about the "$rel_acq_abs$" variable here. How is this coefficient identified? isn't it impossible to have repeat deals for which $per_rel_acq_abs$ is 1 by construction? }

\subsubsection{Exploiting exogenous variation in banker turnover from changes in salary levels}

In this section, we exploit a plausibly exogenous variation to the composition of bank's pool of bankers as a shock to their personal relationships to borrowers. One of the major reasons why employees change firms is lack of compensation, and this connection is particularly strong in the financial sector \textcolor{red}{cite anecdotal or papers?} . We therefore hypothesize that bankers which are underpaid relative to their peers are more likely to leave their employers, and switch to banks that are paying more attractive salaries. One concern with salary changes is that salaries are tied to individual performance. Bankers with weak performance, for example,  might be more likely to get paid less and at the same time leave their employer. We therefore exploit the fact that bonus payments in banking consist of two components: a bank-wide performance component, as well as an individual component. Banks allocate bonuses from a general bonus pool. If the bank performs well, the bonus pool is larger and all employees benefit from higher payments. If the bonus pool is smaller, all employees, irrespective of performance, have to accept smaller bonuses.  \textcolor{red}{cite anecdotal or papers?}

While there is no data on the specific compensation of individual employees, we can proxy for the bank-wide portion of salaries by observing the compensation of board members. We argue that , when bank performance goes down, salaries decrease across the entire bank, leading to a exodus of commercial bankers. Importantly, commercial lending is often only a minor part of bank's profits, and hence a drop in overall bank wide profits is unlikely to be related to the performance of the commercial lending unit. \textcolor{red}{cite saidi neuman 2020 they make similar argument for mergers?} 

We estimate a 2SLS model on the bank-year level. In the first stage, we estimate the impact of high bank level salaries, measured as the \% of salaries above the median, on the bank's ability to attract additional bankers. The outcome variable is therefore $\Sigma Rel_acq$, the total number of relationships acquired. In the second stage, we estimate the effect of the instrumented number of relationships acquired on the likelihood of initiating new lending relationships. The results are presented in Table ~\ref{tab:iv}.

\begin{center} - Table~\ref{tab:iv} - \end{center}

Column one shows that paying above the industry median allows banks to attract significantly more bankers. The coefficient estimate of \% \emph{Comp above median} on $\Sigma Rel_acq$ is 3.64, and statistically significant at the 5\% level. Paying above median slaaries therefore helps banks attract more bankers, which suggests that the instrument fulfills the relevancy criterion. \textcolor{blue}{The Kleibergen-Paap F statistic for the first stage is 8.04, which is above the critical Stock and Yogo level, alleviating concerns of a weak instrument.} A bank paying 10\% above the median in any given year therefore gains about 35 new connections to borrowers, which corresponds to hiring about 8 additional bankers. The second stage results, in column (2), show that when we instrument personal relationship acquisitions in this manner, we find the same relationship between personal relationship acquisition and the initiation of new lending relationships: The coefficient estimate on the instrumented $\Sigma Rel_acq$ is 4.72, and statistically significant at the 5\% level. \textcolor{blue}{Can we check the magnitudes here? My interpretation would be that for every relationship acquired, you gain 4.7 initiations? That cannot possibly be correct, right? Do we have to divide by 100? I will just stop writing here, feels like this result might not survive.}

\textcolor{red}{Clearly when we write it like this it's a problem. Some ideas: 1. can we maybe plot total board compensation over time and show it is higher during booms and lower during recessions (to substantiate our claim that compensation falls irrespective of performance?)  2. can we try to address the bank issue? I seem to remember that bank FE result in too little variation left over to run our estimation. Is that also true for the quartiles? Is it true if we use executive compensation rather than all board members? Maybe using the Option value of compensation in combination with only executives and quartiles? 3. Alternatively, if it really doesn't work, we can at least proxy for bank size by controling for total number of clients the bank has (either continously or as quantiles? that won't hold up to too much scrutiny but a favorable ref might like it. )}
%%%%%%%%%%%%%%%%%%%%%%%%%%%%%%%%%%%%%%%%%%%%%%%%%%%%%%%%
%% culture/female
\subsubsection{Gender as a friction linking labor and capital markets}

In this section, we investigate one particular friction that potentially spills over form bankers' labor markets to corporate loan markets, namely the way corporate culture towards gender impacts female bankers and their clients. The financial industry is often accused of being a particularly hostile work environment for women \citep{Jaekel.2016}, 
%\footnote{\url{https://hbr.org/2016/10/why-women-arent-making-it-to-the-top-of-financial-services-firms} %harvard/OW study. maybe disclosure from congressional oversight committee?
and the gender pay gap in financial services is larger than in most industries \citep{BLS2019}. %(CITE\footnote{\url{https://www.adp.com/spark/articles/2019/09/assessing-the-gender-pay-gap-in-the-finance-sector.aspx} \url{https://www.glassdoor.com/research/app/uploads/sites/2/2016/03/Glassdoor-Gender-Pay-Gap-Study-3.pdf} fig4 p23. glassdoor data}). 
We hypothesize that frictions int he labor market from work environment unfriendly to female bankers will induce them to switch employers, and lead to a shift in lenders for their clients. %female bankers will leave institutions when the culture turns less female friendly.
\textcolor{red}{CH: stupid question: are our $discrimination_{t-1}$ variables defined only for those bankers that switch employers? That would induce mechanical relationships, right? Because it is only defined for those that switch? But then we would not be able to estimate this model at all bc there would be a 1-1 mapping between that variable and leaving? Sorry I am confused I am really tired right now. Let's think it through together?	}

We employ two separate proxies for each bank's climate towards female bankers. Our first proxy builds on prior studies that have shown that female directors improve the climate for female employees \citep{Bilimoria2006}. We therefore measure the representation of women on each bank's board of directors as a measure of female friendly culture, and define the indicator $No~female~director_{t-1}$ to mark those bankers who worked, in the previous year, at an employer without female board members.

Next, we obtain data on labor lawsuits form a non profit organization, the \emph{Good Jobs First} initiative.\footnote{Available at \url{https://www.goodjobsfirst.org/sites/default/files/docs/pdfs/BigBusinessBias.pdf}} 
We then create two indicator variables measuring bank's lawsuit propensity. First, \emph{Empl. discrimination$_{t-1}$}, an indicator that takes the value one for each bank against which at least one general employment discrimination lawsuits was brought in the past. These include lawsuits for discrimination based on age, disability, religion, national origin, and sexual orientation. Second, \emph{Gender discrimination}, a more focused indicator of whether a banks has been the target of a gender discrimination lawsuit, including lawsuits on sexual harassment, pregnancy and gender discrimination. 


%https://www.jstor.org/stable/40604524?seq=1#metadata_info_tab_contents

Table \ref{tab:banker_discrimination} presents results of regressions of our measure $Banker~hired$, an indicator in the first year that a banker moves to a new bank, on our measures of female friendly climate interacted with an indicator if a banker is female. All regressions include fixed effects controlling for each bankers old and new banks. We control for time varying propensity to move banks using year fixed effects. Columns 2 and 4 feature additional controls for banker tenure, number of clients and deals, and total number of lawsuits by bank. 
 
\begin{center} - Table~\ref{tab:banker_discrimination} - \end{center}

In columns 1 and 2 of  Table~\ref{tab:main_init_female}, we estimate the impact of female board representation on the likelihood of female bankers switching to new employers. Consistent with a lack of female board members being the result of, or creating, a less supportive environment for female bankers, we find that the coefficient estimate on the interaction of  $Female~director_{t-1} \times Female$ is 3.54,and significant at the 5\% level. These results imply that female bankers were more likely to leave banks with smalelr female board representation. 

In column 3, we investigate whether gender discrimination lawsuits have an impact on female bankers. We indeed find that the coefficient on $Gender discrimination_{t-1} \times Female$ is 4.95, and statistically significant at the 5\% level. Female bankers are therefore almost 5\% more likely to leave a bank facing gender discrimination lawsuits as are their male colleagues. This is a sizable effect that represents a three-fold increase compared to the sample mean. This result stays economically and statistically almost unchanged when we add additional banker level controls in column 4.

%In column 3, we test for a differential affect between more or less senior female bankers. More senior bankers might have longer exposure to the climate and might therefore react differently, either because by selection they have revealed their ability to thrive in a hostile work environment. While the main coefficient on 1(Lawsuit~gender)$\times$ 1(female) remains very stable in this specification, the coefficient on the triple interaction 1(Lawsuit~gender)$\times$ 1(female) $\times$ 1(Tenure>5years), while negative, is statistically insignificant. 
In column 5, we run a placebo test that investigates if \emph{general} labor lawsuits that are not gender discrimination related have a disproportionate effect on female bankers. Indeed, the interaction term  $Empl.~discrimination_{t-1} \times Female$, is negative and  not statistically significant. Female bankers are therefore not more likely than their male counterparts to leave banks faced with general labor lawsuits - they only react differently for specific lawsuits related to gender discrimination. 

These results underscore that the climate for female employees at banks can induce significant shifts in the workforce of bankers. In our next set of tests, we investigate if this friction of work place culture spills over into capital market. Specifically, we investigate if female bankers are able to  transfer bank-borrower relationships for their clients to their new employer to the same degree as their male colleagues. We therefore repeat the analysis of Table \ref{tab:init} and add an interaction term between Rel\_acq and Female that captures whether female bankers exhibit a differential propensity to move clients to new employers. 


\begin{center} - Table~\ref{tab:main_init_female} - \end{center}

In Table~\ref{tab:main_init_female} we find that, across all specifications and variations of the definition of Relationship acquisition, the coefficient estimate on this interaction term is positive and statistically significant, implying that female bankers are at least as likely as their male counterparts to move clients with them to new banks. 

In our final result, we investigate if there is a difference in the performance of female bankers relative to their male colleagues depending on the culture of their employer. a large literature in labor economics \textcolor{red}{cite? } finds that performance of female bankers suffers in the presence of a less female friendly work environment. In Table \ref{tab:female_perf}  we investigate if a similar pattern holds in our sample. We estimate the  models from Table~\ref{tab:main_init_female}  for the entire sample, and separately for the sample of banks with- and without gender discrimination lawsuits. 

\begin{center} - Table~\ref{tab:female_perf} - \end{center}

The results suggest that female bankers are significantly more productive in the work environment of banks that are not subject to gender discrimination lawsuits both for initiating new banking relationships and generating deal volume. This evidence is consistent with a link between the work environment and performance. Overall, the results in this section support the idea that frictions in the labor market for commercial bankers spill over into the formation of new banking relationships in the capital market. 


%Finally, we investigate whether, conditional on the initiation of a relationship, female bankers' clients create similar amounts of deal volume compared to their male colleagues.
%
%\begin{center} - Table~\ref{tab:main_init_female_vol} - \end{center}
%
%Table \ref{tab:main_init_female_vol} repeats the analysis of Table \ref{tab:main_dealsize}, adding an interaction with our Female indicator. We find indeed that across all specifications, female bankers' clients create at least as much deal volume as their male colleagues'. 



% %\subsection{Sources of variation} \label{sec:variation}



% %In our most stringent specification, we include fixed effects for firm-bank pairs, bank-year, and firm-year. Thus, we exploit variation in the likelihood of initiating a new relationship from three separate sources. That is, we exploit variation in these variables 1) within firm-bank pairs across time, 2) within bank-year across firms, and 3) within firm across banks. 



% %Our measure of deal volume follows the same pattern. The panel also shows that there is significant variation in the previous three variables within firm and within year. Overall, there is more than sufficient variation in the data to expect our specifications to have power.




% \subsection{Discussion and alternative explanations}

%It could be that not only the size of the client portfolio matters, but also its distribution across bankers. For instance, a bank might want to acquire the relationships of the best-connected banker. Alternatively, a very successful banker might feel that in order to continue extending its client portfolio, she will need to switch bank. In Panel A of Appendix Table~\ref{atab:banker_client_conc} we show regressions of the switching indicator on the logarithm of the lagged rank that the bankers have within the bank in terms of their client portfolios.\footnote{We rank bankers every year according to their total number of clients/deals at a bank while accounting for draws.} Columns (1) and (2) show that the relative standing within a bank does \emph{not} influence the banker's probability of switching. In columns (3) and (4) we make sure that this finding is not driven by non-linearities. To do so we construct indicator variables for the quartiles of the distribution of number of clients/deals within each bank-year pair. Again, we do not find any significant result. In columns (5) and (6) we are able to confirm our previous finding, that personal relationships to large clients are particularly important in determining the switching probability: If the rank of banker in terms of her relationships with large clients increases by one standard deviation (1.60), the probability of switching becomes 0.25\% larger. The overall result is very similar in Panel B, where we focus on the distribution of deals instead of clients. 

