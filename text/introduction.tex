
%%%%%%%%%%%%%%%%%%%%%%%%%%%%%%%%%%%%%%%%%%%%%%
%MOTIVATION


Bank loans are a primary source of capital for firms, and the choice of lender has major repercussions for borrowers. On the positive side, forming lasting lending relationships impacts both the availability and pricing of credit.\footnote{See, for example, \citep{Diamond.1984, Sharpe.1990, Diamond.1991, Rajan.1992, Petersen.1994, Berger.1995, Bharath.2007, Ioannidou.2010}} On the other hand, being connected to a lender also exposes borrowers to adverse shocks on the lender level. When the lender suffers losses, it can ration credit to its relationship borrowers.\footnote{For example, \citep{Holmstrom1997, khwaja2008tracing, leary2009bank, Ivashina.2010, Chava2011, Lin2013, ChodorowReich.2014}} Lending relationships therefore impact borrowers and transmit shocks from the financial to the real sector. 

While it is well understood that lending relationships have a large impact on borrowers, %XXX add theory and empirics in footnote
there is much less evidence on how these relationships are formed, and what economic forces drive the matching of banks and borrowers. 
In particular, why would borrowers give up on the inherent value in an established lending relationship by switching banks? In this paper, we take a step towards answering this important question by studying the interaction between the labor market for commercial bankers and capital markets for borrower. We hypothesize that these bankers, who are the bank's source of soft information about borrowers \citep{Berger2002} play a key role in matching borrowers to banks and that frictions in the labor market for bankers have consequences in capital markets. 


Consistent with banks recognizing the ability of bankers to bring additional business, we find a strong positive relationship between the value of a banker's portfolio of relationships and the likelihood that the banker gets poached by a competing bank. A one standard deviation increase in the number of clients in a banker's portfolio (4.29) increases the chance that this banker switches to a competing bank by 0.65\%, %(0.15\%*4.29), or 1.58
corresponding to a sizable 40\% increase compared to the unconditional likelihood. Not all relationships are equally valuable. Larger clients produce more business and, therefore, revenue than smaller ones. Consistent with this intuition, we show that the number of larger clients is more important in predicting which bankers get poached than the number of small clients. These results are consistent with a wide anecdotal evidence of the dynamics between the labor market for bankers and relationship lending.\footnote{Appendix \ref{app:anecdotal} provides extensive examples.}

If bankers use their existing relationship with borrowers to lure them along to new lenders, their ability to do so should be stronger if they are the sole contact person with the borrower. When we estimate the differential effect for clients that are exclusive to a single banker as opposed to those with multiple points of contact, we find that only clients that are exclusivele to a single banker predict banker turnover, and are subsequently following to the new bank. All these result are even more pronounced when we measure the value of a banker's portfolio as the number of prior deals (instead of the number of clients), with a one standard deviation increase in the number of deals by a banker's portfolio of clients being associated with a doubling of the unconditional rate of departure to a competing bank. %0.10*17.43=1.75

%Summary
We next investigate if bankers indeed succeed in bringing their previous clients with them. We find that personal relationships between bankers and firms are a key factor in matching lenders to borrowers. After a commercial banker with an established lending relationship with a borrower switches from one bank to another, the incidence of the borrower following the banker to the new employer increases by 14\% every year. This effect is economically sizable, representing an increase by a factor of almost 3 compared to the unconditional sample average. %calc: coefficient is 14 mean is 5.2 


Importantly, these results hold under tight controls including borrower-bank fixed effects which absorb all borrower-bank characteristics that are stable through time, such as industry, size, or corporate culture. Our results are therefore not driven by banks expanding both business and pool of bankers, business cycle fluctuations, or the general fit between a bank and a borrower. Instead, within-borrower-bank changes in having a personal relationship through a banker. In addition, bank-year and borrower-year fixed effects control for lender and borrower time specific trends in initiation new relationships. These controls rule out a wide range of alternative explanations for our findings, such as a lender expanding and both hiring additional employees and initiating new lending relationships. To the best of our knowledge, this paper is the first to document the importance of individual bankers in creating these important bank-borrower ties. 

The relationships a lender can acquire by poaching a banker from a competitor have significant commercial value. After a banker switches to a new lender, that lender's deal volume with the banker's relationship clients increases by 35\%. %, or \$XXXX at the median deal size. 
Importantly, these new clients do not just bring lending business, but also other types of underwriting services. We find that after a banker moves, the banker's former borrower also issues new bonds with the new lender. %needs to check SEO
The ability to cross sell other underwriting mandates shows that lending relationships anchor the banking relationship more widely, creating additional benefits for banks. from acquiring new borrowers.
The relationships acquired with a new banker are also long lasting. We find that lenders produce significant deal flow with borrowers that switched with a relationship banker for years after the initial deal. 

In our final set of results, we investigate how one particular friction in labor markets spills over into capital markets. Using gender discrimination lawsuits filed against banks and the absence of female directors as proxies for less female friendly cultures, we find that banks with a less female friendly culture lose female bankers, and their clients. 

Overall, our results show that there are important interactions between the labor market of individual bankers and the bank lending market. %Our results also help explain why bankers are being paid substantial 


%Literature
We contribute to a number of papers on the importance of individual bankers in the US syndicated loan market \citep{Herpfer.2018, Bushman2019}. These papers show that bankers impact loans both through time invariant preferences or styles, as well as through the personal relationships they form with borrowers. These lending relationships give bankers a deep understanding of borrowers, and allow them to match borrowers to collaborate in joint ventures \citep{Frattaroli2019}. Consistent with these bankers having a sizable impact on lending decisions, there is evidence that they get punished after one of their borrowers defaults \citep{Gao.2018b}.\footnote{Individuals making connections between their networks is commonly observed in the business world outside the context of banking. \citet{Hacamo2019}, for example, show that managers allow their employers to hire talent from their personal network.}

Our paper is further related to the literature on the formation of banking relationships. Borrowers tend to switch lenders more frequently when they are smaller \citep{Ongena.2001}, and obtaining loans from a new lender is associated with favorable credit conditions \citep{Gopalan2011} although these tend to worsen over time as lenders gain private information about borrowers, creating a ``hold up'' problem \citep{Ioannidou.2010, schenone2010lending, Sharpe.1990, Rajan.1992, Thadden.2004} . Borrowers can mitigate this holdup issue by maintaining multiple lending relationships \citep{Farinha.2002}. \citet{Karolyi.2017} finds that borrowers often switch lenders after the personal relationship between their executives and lenders is broken. In the paper closest to ours, \citet{Schwert2018} shows that banks and borrowers match on observable criteria and that more bank relying borrowers match with more well capitalized banks. We add to his findings by showing that, apart from object firm and bank characteristics, there is a substantial human factor in the formation of bank borrower relationships. 

%cite JJ labor frictions

% Berger, Miller, Petersen, Rajan and Stein (2005) shows it is the number of branches that matter for borrower retention, not bank asset size.  >> Use as starting point of why bankers instead of banks might matter 

%Results

