%%%
%%%
Lending relationships are a key source of capital to the economy, and the lending relationships formed with banks deeply impact borrowers. While the importance of these relationships is well understood, our paper sheds light on one specific mechanisms through which the matching between firms and banks occurs. We use a novel dataset that identifies individual bankers and follows them through time and across banks. We are thus able to show (1) which factors influence banker turnover and crucially (2) whether the personal relationships that banks acquire by employing new bankers translate into new business opportunities.

As for (1), we find that bankers with client portfolios that are more valuable are more likely to switch to a new bank, consistent with the idea that banks recognize the ability of bankers to bring new business. A one standard deviation increase in the number of clients in a banker’s portfolio (4.29) increases the chance that this banker switches to a competing bank 40\% compared to the unconditional probability. Intuitively, some client relationships are more valuable than others. Consistent with this, we find that large clients, i.e., those that are likely to generate more business, and single-contact clients, i.e., those that are more likely to have a strong relationship with the banker, play a particularly important role in determining the turnover probability of bankers.

We find strong support for (2): First, when banks acquire a personal relationship to a client by scooping a banker, they increase they substantially increase the likelihood of initiating a new business relation with this firm. Second, these new relationships are accountable for a boost in deal volume compared not only to the banks' existing clients but also relative to new relationships that the bank might form without having a personal contact. Finally, the additional deal volume we document extends beyond syndicated lending to cover bond underwriting and the effect is long lasting.

Having established the importance of individual bankers in shaping banking relationships, we document that frictions from the labor market of corporate bankers spill over into the capital market for borrowers. Specifically, we show that female bankers switch between lenders in a predictable fashion, leaving banks with less female friendly work environments for other banks, in which they subsequently outperform. In the process, these bankers lead to a re-arranging of lending relationships for their borrowers.

Taken together our results show that there are important interactions between the labor market of individual bankers and the bank lending market. Our results highlight the importance of bankers in enabling the formation of banking relationships and uncover the bankers' characteristics that are material to explain banker turnover. Future work could identify other types of labor market frictions, and further identify the effects of these frictions on borrower performance and loan outcomes.